
% ==============================================================================
\chapter{Document layout}
% ==============================================================================

\begin{frontmatter}

\begin{abstract}
This chapter describes the document layout, including the compilation instructions.
\end{abstract}

\keywords{\LaTeX; class; article}

\begin{remark}
  T.W.J. de Geus
\end{remark}

\end{frontmatter}

% ==============================================================================
\section{Preamble}
% ==============================================================================

% ==============================================================================
\subsection{Introduction}
% ==============================================================================

By default most of the standard \LaTeX-packages are loaded. Any of these packages can be re-loaded, with other defaults, without problems. In addition the title and author can be specified; see below.

% ==============================================================================
\subsection{Load class}
% ==============================================================================

To load the class use
\begin{verbatim}
  \documentclass{goose-thesis}
\end{verbatim}
%
To use customized fonts, the documents has to be compiled using XeLaTeX. For example:
\begin{verbatim}
  %!TEX program = XeLaTeX
  \documentclass[garamond]{goose-thesis}
\end{verbatim}
%
The following fonts are available:
%
\begin{itemize}
  %
  \item \texttt{garamond}
  \item \texttt{times}
  \item \texttt{verdana}
  %
\end{itemize}
%
%
Furthermore the following options are available
%
\begin{itemize}
  %
  \item \texttt{narrow}: widen the margins of the page, useful during the review process;
  \item \texttt{doublespacing}: set the line-spacing to double, useful during the review process.
  \item \texttt{namecite}: use names instead of number of citations.
  \item \texttt{sectionbib}: include the bibliography at the end of each chapter.
  %
\end{itemize}
%

% ==============================================================================
\subsection{Title and author}
% ==============================================================================

%
\begin{itemize}
%
\item The \textit{title} is specified using
\begin{verbatim}
  \title{...}
\end{verbatim}
%
\item The \textit{author} is specified using
\begin{verbatim}
  \author{...}
\end{verbatim}
%
\item Additionally one could decide to change the author of the PDF-document
\begin{verbatim}
  \hypersetup{pdfauthor={...}}
\end{verbatim}
%
\end{itemize}
%

% ==============================================================================
\section{Main text}
% ==============================================================================

In the simplest form the thesis will have one bibliography at the end of the documents. It also possible to include a bibliography at the end of each chapter, see Section~\ref{sec:sectionbib}.

% ==============================================================================
\subsection{Structure}
% ==============================================================================

The thesis comprises of a main \TeX-files and \TeX-files for each chapter. Furthermore a \texttt{Makefile} can be used to gather the compilation instructions. The suggested structure is as follows
%
\begin{verbatim}
  main.tex
  example_chapter1.tex
  example_chapter2.tex
  ...
  library.bib
  figures/
\end{verbatim}
%

% ==============================================================================
\subsection{Main document}
% ==============================================================================

The layout is as follows
\begin{mdframed}
\begin{verbatim}
\documentclass[options]{goose-thesis}

\title{...}

\author{...}

...

\begin{document}

  \maketitle

  \setcounter{tocdepth}{0}
  \tableofcontents

  \cleardoublepage
  
% ==============================================================================
\chapter{Document layout}
% ==============================================================================

\begin{frontmatter}

\begin{abstract}
This chapter describes the document layout, including the compilation instructions.
\end{abstract}

\keywords{\LaTeX; class; article}

\begin{remark}
  T.W.J. de Geus
\end{remark}

\end{frontmatter}

% ==============================================================================
\section{Preamble}
% ==============================================================================

% ==============================================================================
\subsection{Introduction}
% ==============================================================================

By default most of the standard \LaTeX-packages are loaded. Any of these packages can be re-loaded, with other defaults, without problems. In addition the title and author can be specified; see below.

% ==============================================================================
\subsection{Load class}
% ==============================================================================

To load the class use
\begin{verbatim}
  \documentclass{goose-thesis}
\end{verbatim}
%
To use customized fonts, the documents has to be compiled using XeLaTeX. For example:
\begin{verbatim}
  %!TEX program = XeLaTeX
  \documentclass[garamond]{goose-thesis}
\end{verbatim}
%
The following fonts are available:
%
\begin{itemize}
  %
  \item \texttt{garamond}
  \item \texttt{times}
  \item \texttt{verdana}
  %
\end{itemize}
%
%
Furthermore the following options are available
%
\begin{itemize}
  %
  \item \texttt{narrow}: widen the margins of the page, useful during the review process;
  \item \texttt{doublespacing}: set the line-spacing to double, useful during the review process.
  \item \texttt{namecite}: use names instead of number of citations.
  \item \texttt{sectionbib}: include the bibliography at the end of each chapter.
  %
\end{itemize}
%

% ==============================================================================
\subsection{Title and author}
% ==============================================================================

%
\begin{itemize}
%
\item The \textit{title} is specified using
\begin{verbatim}
  \title{...}
\end{verbatim}
%
\item The \textit{author} is specified using
\begin{verbatim}
  \author{...}
\end{verbatim}
%
\item Additionally one could decide to change the author of the PDF-document
\begin{verbatim}
  \hypersetup{pdfauthor={...}}
\end{verbatim}
%
\end{itemize}
%

% ==============================================================================
\section{Main text}
% ==============================================================================

In the simplest form the thesis will have one bibliography at the end of the documents. It also possible to include a bibliography at the end of each chapter, see Section~\ref{sec:sectionbib}.

% ==============================================================================
\subsection{Structure}
% ==============================================================================

The thesis comprises of a main \TeX-files and \TeX-files for each chapter. Furthermore a \texttt{Makefile} can be used to gather the compilation instructions. The suggested structure is as follows
%
\begin{verbatim}
  main.tex
  example_chapter1.tex
  example_chapter2.tex
  ...
  library.bib
  figures/
\end{verbatim}
%

% ==============================================================================
\subsection{Main document}
% ==============================================================================

The layout is as follows
\begin{mdframed}
\begin{verbatim}
\documentclass[options]{goose-thesis}

\title{...}

\author{...}

...

\begin{document}

  \maketitle

  \setcounter{tocdepth}{0}
  \tableofcontents

  \cleardoublepage
  
% ==============================================================================
\chapter{Document layout}
% ==============================================================================

\begin{frontmatter}

\begin{abstract}
This chapter describes the document layout, including the compilation instructions.
\end{abstract}

\keywords{\LaTeX; class; article}

\begin{remark}
  T.W.J. de Geus
\end{remark}

\end{frontmatter}

% ==============================================================================
\section{Preamble}
% ==============================================================================

% ==============================================================================
\subsection{Introduction}
% ==============================================================================

By default most of the standard \LaTeX-packages are loaded. Any of these packages can be re-loaded, with other defaults, without problems. In addition the title and author can be specified; see below.

% ==============================================================================
\subsection{Load class}
% ==============================================================================

To load the class use
\begin{verbatim}
  \documentclass{goose-thesis}
\end{verbatim}
%
To use customized fonts, the documents has to be compiled using XeLaTeX. For example:
\begin{verbatim}
  %!TEX program = XeLaTeX
  \documentclass[garamond]{goose-thesis}
\end{verbatim}
%
The following fonts are available:
%
\begin{itemize}
  %
  \item \texttt{garamond}
  \item \texttt{times}
  \item \texttt{verdana}
  %
\end{itemize}
%
%
Furthermore the following options are available
%
\begin{itemize}
  %
  \item \texttt{narrow}: widen the margins of the page, useful during the review process;
  \item \texttt{doublespacing}: set the line-spacing to double, useful during the review process.
  \item \texttt{namecite}: use names instead of number of citations.
  \item \texttt{sectionbib}: include the bibliography at the end of each chapter.
  %
\end{itemize}
%

% ==============================================================================
\subsection{Title and author}
% ==============================================================================

%
\begin{itemize}
%
\item The \textit{title} is specified using
\begin{verbatim}
  \title{...}
\end{verbatim}
%
\item The \textit{author} is specified using
\begin{verbatim}
  \author{...}
\end{verbatim}
%
\item Additionally one could decide to change the author of the PDF-document
\begin{verbatim}
  \hypersetup{pdfauthor={...}}
\end{verbatim}
%
\end{itemize}
%

% ==============================================================================
\section{Main text}
% ==============================================================================

In the simplest form the thesis will have one bibliography at the end of the documents. It also possible to include a bibliography at the end of each chapter, see Section~\ref{sec:sectionbib}.

% ==============================================================================
\subsection{Structure}
% ==============================================================================

The thesis comprises of a main \TeX-files and \TeX-files for each chapter. Furthermore a \texttt{Makefile} can be used to gather the compilation instructions. The suggested structure is as follows
%
\begin{verbatim}
  main.tex
  example_chapter1.tex
  example_chapter2.tex
  ...
  library.bib
  figures/
\end{verbatim}
%

% ==============================================================================
\subsection{Main document}
% ==============================================================================

The layout is as follows
\begin{mdframed}
\begin{verbatim}
\documentclass[options]{goose-thesis}

\title{...}

\author{...}

...

\begin{document}

  \maketitle

  \setcounter{tocdepth}{0}
  \tableofcontents

  \cleardoublepage
  
% ==============================================================================
\chapter{Document layout}
% ==============================================================================

\begin{frontmatter}

\begin{abstract}
This chapter describes the document layout, including the compilation instructions.
\end{abstract}

\keywords{\LaTeX; class; article}

\begin{remark}
  T.W.J. de Geus
\end{remark}

\end{frontmatter}

% ==============================================================================
\section{Preamble}
% ==============================================================================

% ==============================================================================
\subsection{Introduction}
% ==============================================================================

By default most of the standard \LaTeX-packages are loaded. Any of these packages can be re-loaded, with other defaults, without problems. In addition the title and author can be specified; see below.

% ==============================================================================
\subsection{Load class}
% ==============================================================================

To load the class use
\begin{verbatim}
  \documentclass{goose-thesis}
\end{verbatim}
%
To use customized fonts, the documents has to be compiled using XeLaTeX. For example:
\begin{verbatim}
  %!TEX program = XeLaTeX
  \documentclass[garamond]{goose-thesis}
\end{verbatim}
%
The following fonts are available:
%
\begin{itemize}
  %
  \item \texttt{garamond}
  \item \texttt{times}
  \item \texttt{verdana}
  %
\end{itemize}
%
%
Furthermore the following options are available
%
\begin{itemize}
  %
  \item \texttt{narrow}: widen the margins of the page, useful during the review process;
  \item \texttt{doublespacing}: set the line-spacing to double, useful during the review process.
  \item \texttt{namecite}: use names instead of number of citations.
  \item \texttt{sectionbib}: include the bibliography at the end of each chapter.
  %
\end{itemize}
%

% ==============================================================================
\subsection{Title and author}
% ==============================================================================

%
\begin{itemize}
%
\item The \textit{title} is specified using
\begin{verbatim}
  \title{...}
\end{verbatim}
%
\item The \textit{author} is specified using
\begin{verbatim}
  \author{...}
\end{verbatim}
%
\item Additionally one could decide to change the author of the PDF-document
\begin{verbatim}
  \hypersetup{pdfauthor={...}}
\end{verbatim}
%
\end{itemize}
%

% ==============================================================================
\section{Main text}
% ==============================================================================

In the simplest form the thesis will have one bibliography at the end of the documents. It also possible to include a bibliography at the end of each chapter, see Section~\ref{sec:sectionbib}.

% ==============================================================================
\subsection{Structure}
% ==============================================================================

The thesis comprises of a main \TeX-files and \TeX-files for each chapter. Furthermore a \texttt{Makefile} can be used to gather the compilation instructions. The suggested structure is as follows
%
\begin{verbatim}
  main.tex
  example_chapter1.tex
  example_chapter2.tex
  ...
  library.bib
  figures/
\end{verbatim}
%

% ==============================================================================
\subsection{Main document}
% ==============================================================================

The layout is as follows
\begin{mdframed}
\begin{verbatim}
\documentclass[options]{goose-thesis}

\title{...}

\author{...}

...

\begin{document}

  \maketitle

  \setcounter{tocdepth}{0}
  \tableofcontents

  \cleardoublepage
  \include{example_chapter1}

  \cleardoublepage
  \include{example_chapter2}

  ...

  \bibliography{...}

\end{document}
\end{verbatim}
\end{mdframed}

Herein the chapters have been included as separate files. Notice that there is a single \verb|\bibliography{...}| entry at the end of the main document. None of the chapters (in the \verb|\include{...}| command) will have such an entry.

% ==============================================================================
\subsection{Chapters}
% ==============================================================================

The layout is as follows
\begin{mdframed}
\begin{verbatim}
\begin{chapter}

\begin{frontmatter}

\begin{abstract}
...
\end{abstract}

\keywords{...}

\begin{remark}
...
\end{remark}

\end{frontmatter}

...



\appendix

...

\end{verbatim}
\end{mdframed}

Notice how each of the chapters has its own appendix.

% ==============================================================================
\subsection{Compilation}
% ==============================================================================

The compilation can be done by compiling the main-file just like any other file.

% ==============================================================================
\section{Main text -- multiple bibliographies}
\label{sec:sectionbib}
% ==============================================================================

% ==============================================================================
\subsection{Structure}
% ==============================================================================

The thesis comprises of a main \TeX-files and \TeX-files for each chapter. Furthermore a \texttt{Makefile} can be used to gather the compilation instructions. The suggested structure is as follows
%
\begin{verbatim}
  main.tex
  example_chapter1.tex
  example_chapter2.tex
  ...
  library.bib
  figures/
\end{verbatim}
%

% ==============================================================================
\subsection{Main document}
% ==============================================================================

The layout is as follows
\begin{mdframed}
\begin{verbatim}
\documentclass[sectionbib,...]{goose-thesis}

\title{...}

\author{...}

...

\begin{document}

  \maketitle

  \setcounter{tocdepth}{0}
  \tableofcontents

  \cleardoublepage
  \include{example_chapter1}

  \cleardoublepage
  \include{example_chapter2}

  ...

\end{document}
\end{verbatim}
\end{mdframed}

Notice that in this case the main file does not have a \verb|\bibliography{...}| command. Rather, each of the chapter contains this command (unless of course that are no citations in that chapter). Also notice that the \verb|sectionbib| option has been used.

% ==============================================================================
\subsection{Chapters}
% ==============================================================================

The layout is as follows
\begin{mdframed}
\begin{verbatim}
\begin{chapter}

\begin{frontmatter}

\begin{abstract}
...
\end{abstract}

\keywords{...}

\begin{remark}
...
\end{remark}

\end{frontmatter}

...

\bibliography{...}

\appendix

...

\end{verbatim}
\end{mdframed}

Notice the \verb|\bibliography{...}| command at the end of the chapter.

% ==============================================================================
\subsection{Compilation}
% ==============================================================================

In this case the compilation is a bit more involved, as several bibliographies have to be created. For this example the steps are included in the following \texttt{Makefile}:

\begin{mdframed}
\begin{verbatim}
all:
  xelatex -interaction=nonstopmode example.tex
  xelatex -interaction=nonstopmode example.tex
  bibtex example_chapter1
  bibtex example_chapter2
  xelatex -interaction=nonstopmode example.tex
  xelatex -interaction=nonstopmode example.tex

clean:
  rm *.aux *.bbl *.log *.out *.pdf *.toc *.blg *.fls *.fdb_latexmk
\end{verbatim}
\end{mdframed}

% ==============================================================================
\section{Citations}
% ==============================================================================

Citations and references are handled using \texttt{natbib}. To cite use
\begin{verbatim}
  \citep{...}  (or \cite{...})
  \citet{...}
\end{verbatim}
The former only inserted a citation as number. For example \citep{geus}. The latter also includes the name(s) of the author(s). For example \citet{geus,wiki}.

The bibliography information is stored in a \texttt{bib}-file, which is included using
\begin{verbatim}
  \bibliography{...}
\end{verbatim}
This command creates a chapter section ``References'' with the bibliography. By default number citations are used, in which the references appear in the order in which they were cited. In the case that the \verb|namecite| option is used, the citations appear in names in alphabetical order.

Note that a large part of the formatting of Bib\TeX~depends on the formatting of the \texttt{bib}-file. A Python-script \texttt{bibparse} is available to automatically clean-up the formatting of the \texttt{bib}-file. An updated \texttt{unsrtnat.bst} is available that includes the \texttt{eprint} field.

% ==============================================================================
\bibliography{example_refs}
% ==============================================================================

\appendix

% ==============================================================================
\section{Some appendix}
% ==============================================================================


  \cleardoublepage
  
\chapter{Another chapter}

\begin{frontmatter}

\begin{abstract}
This is another chapter.
\end{abstract}

\end{frontmatter}

\section{With a section}

And a citation \citep{wiki}.

\bibliography{example_refs}

\appendix

\section{And an appendix}


  ...

  \bibliography{...}

\end{document}
\end{verbatim}
\end{mdframed}

Herein the chapters have been included as separate files. Notice that there is a single \verb|\bibliography{...}| entry at the end of the main document. None of the chapters (in the \verb|\include{...}| command) will have such an entry.

% ==============================================================================
\subsection{Chapters}
% ==============================================================================

The layout is as follows
\begin{mdframed}
\begin{verbatim}
\begin{chapter}

\begin{frontmatter}

\begin{abstract}
...
\end{abstract}

\keywords{...}

\begin{remark}
...
\end{remark}

\end{frontmatter}

...



\appendix

...

\end{verbatim}
\end{mdframed}

Notice how each of the chapters has its own appendix.

% ==============================================================================
\subsection{Compilation}
% ==============================================================================

The compilation can be done by compiling the main-file just like any other file.

% ==============================================================================
\section{Main text -- multiple bibliographies}
\label{sec:sectionbib}
% ==============================================================================

% ==============================================================================
\subsection{Structure}
% ==============================================================================

The thesis comprises of a main \TeX-files and \TeX-files for each chapter. Furthermore a \texttt{Makefile} can be used to gather the compilation instructions. The suggested structure is as follows
%
\begin{verbatim}
  main.tex
  example_chapter1.tex
  example_chapter2.tex
  ...
  library.bib
  figures/
\end{verbatim}
%

% ==============================================================================
\subsection{Main document}
% ==============================================================================

The layout is as follows
\begin{mdframed}
\begin{verbatim}
\documentclass[sectionbib,...]{goose-thesis}

\title{...}

\author{...}

...

\begin{document}

  \maketitle

  \setcounter{tocdepth}{0}
  \tableofcontents

  \cleardoublepage
  
% ==============================================================================
\chapter{Document layout}
% ==============================================================================

\begin{frontmatter}

\begin{abstract}
This chapter describes the document layout, including the compilation instructions.
\end{abstract}

\keywords{\LaTeX; class; article}

\begin{remark}
  T.W.J. de Geus
\end{remark}

\end{frontmatter}

% ==============================================================================
\section{Preamble}
% ==============================================================================

% ==============================================================================
\subsection{Introduction}
% ==============================================================================

By default most of the standard \LaTeX-packages are loaded. Any of these packages can be re-loaded, with other defaults, without problems. In addition the title and author can be specified; see below.

% ==============================================================================
\subsection{Load class}
% ==============================================================================

To load the class use
\begin{verbatim}
  \documentclass{goose-thesis}
\end{verbatim}
%
To use customized fonts, the documents has to be compiled using XeLaTeX. For example:
\begin{verbatim}
  %!TEX program = XeLaTeX
  \documentclass[garamond]{goose-thesis}
\end{verbatim}
%
The following fonts are available:
%
\begin{itemize}
  %
  \item \texttt{garamond}
  \item \texttt{times}
  \item \texttt{verdana}
  %
\end{itemize}
%
%
Furthermore the following options are available
%
\begin{itemize}
  %
  \item \texttt{narrow}: widen the margins of the page, useful during the review process;
  \item \texttt{doublespacing}: set the line-spacing to double, useful during the review process.
  \item \texttt{namecite}: use names instead of number of citations.
  \item \texttt{sectionbib}: include the bibliography at the end of each chapter.
  %
\end{itemize}
%

% ==============================================================================
\subsection{Title and author}
% ==============================================================================

%
\begin{itemize}
%
\item The \textit{title} is specified using
\begin{verbatim}
  \title{...}
\end{verbatim}
%
\item The \textit{author} is specified using
\begin{verbatim}
  \author{...}
\end{verbatim}
%
\item Additionally one could decide to change the author of the PDF-document
\begin{verbatim}
  \hypersetup{pdfauthor={...}}
\end{verbatim}
%
\end{itemize}
%

% ==============================================================================
\section{Main text}
% ==============================================================================

In the simplest form the thesis will have one bibliography at the end of the documents. It also possible to include a bibliography at the end of each chapter, see Section~\ref{sec:sectionbib}.

% ==============================================================================
\subsection{Structure}
% ==============================================================================

The thesis comprises of a main \TeX-files and \TeX-files for each chapter. Furthermore a \texttt{Makefile} can be used to gather the compilation instructions. The suggested structure is as follows
%
\begin{verbatim}
  main.tex
  example_chapter1.tex
  example_chapter2.tex
  ...
  library.bib
  figures/
\end{verbatim}
%

% ==============================================================================
\subsection{Main document}
% ==============================================================================

The layout is as follows
\begin{mdframed}
\begin{verbatim}
\documentclass[options]{goose-thesis}

\title{...}

\author{...}

...

\begin{document}

  \maketitle

  \setcounter{tocdepth}{0}
  \tableofcontents

  \cleardoublepage
  \include{example_chapter1}

  \cleardoublepage
  \include{example_chapter2}

  ...

  \bibliography{...}

\end{document}
\end{verbatim}
\end{mdframed}

Herein the chapters have been included as separate files. Notice that there is a single \verb|\bibliography{...}| entry at the end of the main document. None of the chapters (in the \verb|\include{...}| command) will have such an entry.

% ==============================================================================
\subsection{Chapters}
% ==============================================================================

The layout is as follows
\begin{mdframed}
\begin{verbatim}
\begin{chapter}

\begin{frontmatter}

\begin{abstract}
...
\end{abstract}

\keywords{...}

\begin{remark}
...
\end{remark}

\end{frontmatter}

...



\appendix

...

\end{verbatim}
\end{mdframed}

Notice how each of the chapters has its own appendix.

% ==============================================================================
\subsection{Compilation}
% ==============================================================================

The compilation can be done by compiling the main-file just like any other file.

% ==============================================================================
\section{Main text -- multiple bibliographies}
\label{sec:sectionbib}
% ==============================================================================

% ==============================================================================
\subsection{Structure}
% ==============================================================================

The thesis comprises of a main \TeX-files and \TeX-files for each chapter. Furthermore a \texttt{Makefile} can be used to gather the compilation instructions. The suggested structure is as follows
%
\begin{verbatim}
  main.tex
  example_chapter1.tex
  example_chapter2.tex
  ...
  library.bib
  figures/
\end{verbatim}
%

% ==============================================================================
\subsection{Main document}
% ==============================================================================

The layout is as follows
\begin{mdframed}
\begin{verbatim}
\documentclass[sectionbib,...]{goose-thesis}

\title{...}

\author{...}

...

\begin{document}

  \maketitle

  \setcounter{tocdepth}{0}
  \tableofcontents

  \cleardoublepage
  \include{example_chapter1}

  \cleardoublepage
  \include{example_chapter2}

  ...

\end{document}
\end{verbatim}
\end{mdframed}

Notice that in this case the main file does not have a \verb|\bibliography{...}| command. Rather, each of the chapter contains this command (unless of course that are no citations in that chapter). Also notice that the \verb|sectionbib| option has been used.

% ==============================================================================
\subsection{Chapters}
% ==============================================================================

The layout is as follows
\begin{mdframed}
\begin{verbatim}
\begin{chapter}

\begin{frontmatter}

\begin{abstract}
...
\end{abstract}

\keywords{...}

\begin{remark}
...
\end{remark}

\end{frontmatter}

...

\bibliography{...}

\appendix

...

\end{verbatim}
\end{mdframed}

Notice the \verb|\bibliography{...}| command at the end of the chapter.

% ==============================================================================
\subsection{Compilation}
% ==============================================================================

In this case the compilation is a bit more involved, as several bibliographies have to be created. For this example the steps are included in the following \texttt{Makefile}:

\begin{mdframed}
\begin{verbatim}
all:
  xelatex -interaction=nonstopmode example.tex
  xelatex -interaction=nonstopmode example.tex
  bibtex example_chapter1
  bibtex example_chapter2
  xelatex -interaction=nonstopmode example.tex
  xelatex -interaction=nonstopmode example.tex

clean:
  rm *.aux *.bbl *.log *.out *.pdf *.toc *.blg *.fls *.fdb_latexmk
\end{verbatim}
\end{mdframed}

% ==============================================================================
\section{Citations}
% ==============================================================================

Citations and references are handled using \texttt{natbib}. To cite use
\begin{verbatim}
  \citep{...}  (or \cite{...})
  \citet{...}
\end{verbatim}
The former only inserted a citation as number. For example \citep{geus}. The latter also includes the name(s) of the author(s). For example \citet{geus,wiki}.

The bibliography information is stored in a \texttt{bib}-file, which is included using
\begin{verbatim}
  \bibliography{...}
\end{verbatim}
This command creates a chapter section ``References'' with the bibliography. By default number citations are used, in which the references appear in the order in which they were cited. In the case that the \verb|namecite| option is used, the citations appear in names in alphabetical order.

Note that a large part of the formatting of Bib\TeX~depends on the formatting of the \texttt{bib}-file. A Python-script \texttt{bibparse} is available to automatically clean-up the formatting of the \texttt{bib}-file. An updated \texttt{unsrtnat.bst} is available that includes the \texttt{eprint} field.

% ==============================================================================
\bibliography{example_refs}
% ==============================================================================

\appendix

% ==============================================================================
\section{Some appendix}
% ==============================================================================


  \cleardoublepage
  
\chapter{Another chapter}

\begin{frontmatter}

\begin{abstract}
This is another chapter.
\end{abstract}

\end{frontmatter}

\section{With a section}

And a citation \citep{wiki}.

\bibliography{example_refs}

\appendix

\section{And an appendix}


  ...

\end{document}
\end{verbatim}
\end{mdframed}

Notice that in this case the main file does not have a \verb|\bibliography{...}| command. Rather, each of the chapter contains this command (unless of course that are no citations in that chapter). Also notice that the \verb|sectionbib| option has been used.

% ==============================================================================
\subsection{Chapters}
% ==============================================================================

The layout is as follows
\begin{mdframed}
\begin{verbatim}
\begin{chapter}

\begin{frontmatter}

\begin{abstract}
...
\end{abstract}

\keywords{...}

\begin{remark}
...
\end{remark}

\end{frontmatter}

...

\bibliography{...}

\appendix

...

\end{verbatim}
\end{mdframed}

Notice the \verb|\bibliography{...}| command at the end of the chapter.

% ==============================================================================
\subsection{Compilation}
% ==============================================================================

In this case the compilation is a bit more involved, as several bibliographies have to be created. For this example the steps are included in the following \texttt{Makefile}:

\begin{mdframed}
\begin{verbatim}
all:
  xelatex -interaction=nonstopmode example.tex
  xelatex -interaction=nonstopmode example.tex
  bibtex example_chapter1
  bibtex example_chapter2
  xelatex -interaction=nonstopmode example.tex
  xelatex -interaction=nonstopmode example.tex

clean:
  rm *.aux *.bbl *.log *.out *.pdf *.toc *.blg *.fls *.fdb_latexmk
\end{verbatim}
\end{mdframed}

% ==============================================================================
\section{Citations}
% ==============================================================================

Citations and references are handled using \texttt{natbib}. To cite use
\begin{verbatim}
  \citep{...}  (or \cite{...})
  \citet{...}
\end{verbatim}
The former only inserted a citation as number. For example \citep{geus}. The latter also includes the name(s) of the author(s). For example \citet{geus,wiki}.

The bibliography information is stored in a \texttt{bib}-file, which is included using
\begin{verbatim}
  \bibliography{...}
\end{verbatim}
This command creates a chapter section ``References'' with the bibliography. By default number citations are used, in which the references appear in the order in which they were cited. In the case that the \verb|namecite| option is used, the citations appear in names in alphabetical order.

Note that a large part of the formatting of Bib\TeX~depends on the formatting of the \texttt{bib}-file. A Python-script \texttt{bibparse} is available to automatically clean-up the formatting of the \texttt{bib}-file. An updated \texttt{unsrtnat.bst} is available that includes the \texttt{eprint} field.

% ==============================================================================
\bibliography{example_refs}
% ==============================================================================

\appendix

% ==============================================================================
\section{Some appendix}
% ==============================================================================


  \cleardoublepage
  
\chapter{Another chapter}

\begin{frontmatter}

\begin{abstract}
This is another chapter.
\end{abstract}

\end{frontmatter}

\section{With a section}

And a citation \citep{wiki}.

\bibliography{example_refs}

\appendix

\section{And an appendix}


  ...

  \bibliography{...}

\end{document}
\end{verbatim}
\end{mdframed}

Herein the chapters have been included as separate files. Notice that there is a single \verb|\bibliography{...}| entry at the end of the main document. None of the chapters (in the \verb|\include{...}| command) will have such an entry.

% ==============================================================================
\subsection{Chapters}
% ==============================================================================

The layout is as follows
\begin{mdframed}
\begin{verbatim}
\begin{chapter}

\begin{frontmatter}

\begin{abstract}
...
\end{abstract}

\keywords{...}

\begin{remark}
...
\end{remark}

\end{frontmatter}

...



\appendix

...

\end{verbatim}
\end{mdframed}

Notice how each of the chapters has its own appendix.

% ==============================================================================
\subsection{Compilation}
% ==============================================================================

The compilation can be done by compiling the main-file just like any other file.

% ==============================================================================
\section{Main text -- multiple bibliographies}
\label{sec:sectionbib}
% ==============================================================================

% ==============================================================================
\subsection{Structure}
% ==============================================================================

The thesis comprises of a main \TeX-files and \TeX-files for each chapter. Furthermore a \texttt{Makefile} can be used to gather the compilation instructions. The suggested structure is as follows
%
\begin{verbatim}
  main.tex
  example_chapter1.tex
  example_chapter2.tex
  ...
  library.bib
  figures/
\end{verbatim}
%

% ==============================================================================
\subsection{Main document}
% ==============================================================================

The layout is as follows
\begin{mdframed}
\begin{verbatim}
\documentclass[sectionbib,...]{goose-thesis}

\title{...}

\author{...}

...

\begin{document}

  \maketitle

  \setcounter{tocdepth}{0}
  \tableofcontents

  \cleardoublepage
  
% ==============================================================================
\chapter{Document layout}
% ==============================================================================

\begin{frontmatter}

\begin{abstract}
This chapter describes the document layout, including the compilation instructions.
\end{abstract}

\keywords{\LaTeX; class; article}

\begin{remark}
  T.W.J. de Geus
\end{remark}

\end{frontmatter}

% ==============================================================================
\section{Preamble}
% ==============================================================================

% ==============================================================================
\subsection{Introduction}
% ==============================================================================

By default most of the standard \LaTeX-packages are loaded. Any of these packages can be re-loaded, with other defaults, without problems. In addition the title and author can be specified; see below.

% ==============================================================================
\subsection{Load class}
% ==============================================================================

To load the class use
\begin{verbatim}
  \documentclass{goose-thesis}
\end{verbatim}
%
To use customized fonts, the documents has to be compiled using XeLaTeX. For example:
\begin{verbatim}
  %!TEX program = XeLaTeX
  \documentclass[garamond]{goose-thesis}
\end{verbatim}
%
The following fonts are available:
%
\begin{itemize}
  %
  \item \texttt{garamond}
  \item \texttt{times}
  \item \texttt{verdana}
  %
\end{itemize}
%
%
Furthermore the following options are available
%
\begin{itemize}
  %
  \item \texttt{narrow}: widen the margins of the page, useful during the review process;
  \item \texttt{doublespacing}: set the line-spacing to double, useful during the review process.
  \item \texttt{namecite}: use names instead of number of citations.
  \item \texttt{sectionbib}: include the bibliography at the end of each chapter.
  %
\end{itemize}
%

% ==============================================================================
\subsection{Title and author}
% ==============================================================================

%
\begin{itemize}
%
\item The \textit{title} is specified using
\begin{verbatim}
  \title{...}
\end{verbatim}
%
\item The \textit{author} is specified using
\begin{verbatim}
  \author{...}
\end{verbatim}
%
\item Additionally one could decide to change the author of the PDF-document
\begin{verbatim}
  \hypersetup{pdfauthor={...}}
\end{verbatim}
%
\end{itemize}
%

% ==============================================================================
\section{Main text}
% ==============================================================================

In the simplest form the thesis will have one bibliography at the end of the documents. It also possible to include a bibliography at the end of each chapter, see Section~\ref{sec:sectionbib}.

% ==============================================================================
\subsection{Structure}
% ==============================================================================

The thesis comprises of a main \TeX-files and \TeX-files for each chapter. Furthermore a \texttt{Makefile} can be used to gather the compilation instructions. The suggested structure is as follows
%
\begin{verbatim}
  main.tex
  example_chapter1.tex
  example_chapter2.tex
  ...
  library.bib
  figures/
\end{verbatim}
%

% ==============================================================================
\subsection{Main document}
% ==============================================================================

The layout is as follows
\begin{mdframed}
\begin{verbatim}
\documentclass[options]{goose-thesis}

\title{...}

\author{...}

...

\begin{document}

  \maketitle

  \setcounter{tocdepth}{0}
  \tableofcontents

  \cleardoublepage
  
% ==============================================================================
\chapter{Document layout}
% ==============================================================================

\begin{frontmatter}

\begin{abstract}
This chapter describes the document layout, including the compilation instructions.
\end{abstract}

\keywords{\LaTeX; class; article}

\begin{remark}
  T.W.J. de Geus
\end{remark}

\end{frontmatter}

% ==============================================================================
\section{Preamble}
% ==============================================================================

% ==============================================================================
\subsection{Introduction}
% ==============================================================================

By default most of the standard \LaTeX-packages are loaded. Any of these packages can be re-loaded, with other defaults, without problems. In addition the title and author can be specified; see below.

% ==============================================================================
\subsection{Load class}
% ==============================================================================

To load the class use
\begin{verbatim}
  \documentclass{goose-thesis}
\end{verbatim}
%
To use customized fonts, the documents has to be compiled using XeLaTeX. For example:
\begin{verbatim}
  %!TEX program = XeLaTeX
  \documentclass[garamond]{goose-thesis}
\end{verbatim}
%
The following fonts are available:
%
\begin{itemize}
  %
  \item \texttt{garamond}
  \item \texttt{times}
  \item \texttt{verdana}
  %
\end{itemize}
%
%
Furthermore the following options are available
%
\begin{itemize}
  %
  \item \texttt{narrow}: widen the margins of the page, useful during the review process;
  \item \texttt{doublespacing}: set the line-spacing to double, useful during the review process.
  \item \texttt{namecite}: use names instead of number of citations.
  \item \texttt{sectionbib}: include the bibliography at the end of each chapter.
  %
\end{itemize}
%

% ==============================================================================
\subsection{Title and author}
% ==============================================================================

%
\begin{itemize}
%
\item The \textit{title} is specified using
\begin{verbatim}
  \title{...}
\end{verbatim}
%
\item The \textit{author} is specified using
\begin{verbatim}
  \author{...}
\end{verbatim}
%
\item Additionally one could decide to change the author of the PDF-document
\begin{verbatim}
  \hypersetup{pdfauthor={...}}
\end{verbatim}
%
\end{itemize}
%

% ==============================================================================
\section{Main text}
% ==============================================================================

In the simplest form the thesis will have one bibliography at the end of the documents. It also possible to include a bibliography at the end of each chapter, see Section~\ref{sec:sectionbib}.

% ==============================================================================
\subsection{Structure}
% ==============================================================================

The thesis comprises of a main \TeX-files and \TeX-files for each chapter. Furthermore a \texttt{Makefile} can be used to gather the compilation instructions. The suggested structure is as follows
%
\begin{verbatim}
  main.tex
  example_chapter1.tex
  example_chapter2.tex
  ...
  library.bib
  figures/
\end{verbatim}
%

% ==============================================================================
\subsection{Main document}
% ==============================================================================

The layout is as follows
\begin{mdframed}
\begin{verbatim}
\documentclass[options]{goose-thesis}

\title{...}

\author{...}

...

\begin{document}

  \maketitle

  \setcounter{tocdepth}{0}
  \tableofcontents

  \cleardoublepage
  \include{example_chapter1}

  \cleardoublepage
  \include{example_chapter2}

  ...

  \bibliography{...}

\end{document}
\end{verbatim}
\end{mdframed}

Herein the chapters have been included as separate files. Notice that there is a single \verb|\bibliography{...}| entry at the end of the main document. None of the chapters (in the \verb|\include{...}| command) will have such an entry.

% ==============================================================================
\subsection{Chapters}
% ==============================================================================

The layout is as follows
\begin{mdframed}
\begin{verbatim}
\begin{chapter}

\begin{frontmatter}

\begin{abstract}
...
\end{abstract}

\keywords{...}

\begin{remark}
...
\end{remark}

\end{frontmatter}

...



\appendix

...

\end{verbatim}
\end{mdframed}

Notice how each of the chapters has its own appendix.

% ==============================================================================
\subsection{Compilation}
% ==============================================================================

The compilation can be done by compiling the main-file just like any other file.

% ==============================================================================
\section{Main text -- multiple bibliographies}
\label{sec:sectionbib}
% ==============================================================================

% ==============================================================================
\subsection{Structure}
% ==============================================================================

The thesis comprises of a main \TeX-files and \TeX-files for each chapter. Furthermore a \texttt{Makefile} can be used to gather the compilation instructions. The suggested structure is as follows
%
\begin{verbatim}
  main.tex
  example_chapter1.tex
  example_chapter2.tex
  ...
  library.bib
  figures/
\end{verbatim}
%

% ==============================================================================
\subsection{Main document}
% ==============================================================================

The layout is as follows
\begin{mdframed}
\begin{verbatim}
\documentclass[sectionbib,...]{goose-thesis}

\title{...}

\author{...}

...

\begin{document}

  \maketitle

  \setcounter{tocdepth}{0}
  \tableofcontents

  \cleardoublepage
  \include{example_chapter1}

  \cleardoublepage
  \include{example_chapter2}

  ...

\end{document}
\end{verbatim}
\end{mdframed}

Notice that in this case the main file does not have a \verb|\bibliography{...}| command. Rather, each of the chapter contains this command (unless of course that are no citations in that chapter). Also notice that the \verb|sectionbib| option has been used.

% ==============================================================================
\subsection{Chapters}
% ==============================================================================

The layout is as follows
\begin{mdframed}
\begin{verbatim}
\begin{chapter}

\begin{frontmatter}

\begin{abstract}
...
\end{abstract}

\keywords{...}

\begin{remark}
...
\end{remark}

\end{frontmatter}

...

\bibliography{...}

\appendix

...

\end{verbatim}
\end{mdframed}

Notice the \verb|\bibliography{...}| command at the end of the chapter.

% ==============================================================================
\subsection{Compilation}
% ==============================================================================

In this case the compilation is a bit more involved, as several bibliographies have to be created. For this example the steps are included in the following \texttt{Makefile}:

\begin{mdframed}
\begin{verbatim}
all:
  xelatex -interaction=nonstopmode example.tex
  xelatex -interaction=nonstopmode example.tex
  bibtex example_chapter1
  bibtex example_chapter2
  xelatex -interaction=nonstopmode example.tex
  xelatex -interaction=nonstopmode example.tex

clean:
  rm *.aux *.bbl *.log *.out *.pdf *.toc *.blg *.fls *.fdb_latexmk
\end{verbatim}
\end{mdframed}

% ==============================================================================
\section{Citations}
% ==============================================================================

Citations and references are handled using \texttt{natbib}. To cite use
\begin{verbatim}
  \citep{...}  (or \cite{...})
  \citet{...}
\end{verbatim}
The former only inserted a citation as number. For example \citep{geus}. The latter also includes the name(s) of the author(s). For example \citet{geus,wiki}.

The bibliography information is stored in a \texttt{bib}-file, which is included using
\begin{verbatim}
  \bibliography{...}
\end{verbatim}
This command creates a chapter section ``References'' with the bibliography. By default number citations are used, in which the references appear in the order in which they were cited. In the case that the \verb|namecite| option is used, the citations appear in names in alphabetical order.

Note that a large part of the formatting of Bib\TeX~depends on the formatting of the \texttt{bib}-file. A Python-script \texttt{bibparse} is available to automatically clean-up the formatting of the \texttt{bib}-file. An updated \texttt{unsrtnat.bst} is available that includes the \texttt{eprint} field.

% ==============================================================================
\bibliography{example_refs}
% ==============================================================================

\appendix

% ==============================================================================
\section{Some appendix}
% ==============================================================================


  \cleardoublepage
  
\chapter{Another chapter}

\begin{frontmatter}

\begin{abstract}
This is another chapter.
\end{abstract}

\end{frontmatter}

\section{With a section}

And a citation \citep{wiki}.

\bibliography{example_refs}

\appendix

\section{And an appendix}


  ...

  \bibliography{...}

\end{document}
\end{verbatim}
\end{mdframed}

Herein the chapters have been included as separate files. Notice that there is a single \verb|\bibliography{...}| entry at the end of the main document. None of the chapters (in the \verb|\include{...}| command) will have such an entry.

% ==============================================================================
\subsection{Chapters}
% ==============================================================================

The layout is as follows
\begin{mdframed}
\begin{verbatim}
\begin{chapter}

\begin{frontmatter}

\begin{abstract}
...
\end{abstract}

\keywords{...}

\begin{remark}
...
\end{remark}

\end{frontmatter}

...



\appendix

...

\end{verbatim}
\end{mdframed}

Notice how each of the chapters has its own appendix.

% ==============================================================================
\subsection{Compilation}
% ==============================================================================

The compilation can be done by compiling the main-file just like any other file.

% ==============================================================================
\section{Main text -- multiple bibliographies}
\label{sec:sectionbib}
% ==============================================================================

% ==============================================================================
\subsection{Structure}
% ==============================================================================

The thesis comprises of a main \TeX-files and \TeX-files for each chapter. Furthermore a \texttt{Makefile} can be used to gather the compilation instructions. The suggested structure is as follows
%
\begin{verbatim}
  main.tex
  example_chapter1.tex
  example_chapter2.tex
  ...
  library.bib
  figures/
\end{verbatim}
%

% ==============================================================================
\subsection{Main document}
% ==============================================================================

The layout is as follows
\begin{mdframed}
\begin{verbatim}
\documentclass[sectionbib,...]{goose-thesis}

\title{...}

\author{...}

...

\begin{document}

  \maketitle

  \setcounter{tocdepth}{0}
  \tableofcontents

  \cleardoublepage
  
% ==============================================================================
\chapter{Document layout}
% ==============================================================================

\begin{frontmatter}

\begin{abstract}
This chapter describes the document layout, including the compilation instructions.
\end{abstract}

\keywords{\LaTeX; class; article}

\begin{remark}
  T.W.J. de Geus
\end{remark}

\end{frontmatter}

% ==============================================================================
\section{Preamble}
% ==============================================================================

% ==============================================================================
\subsection{Introduction}
% ==============================================================================

By default most of the standard \LaTeX-packages are loaded. Any of these packages can be re-loaded, with other defaults, without problems. In addition the title and author can be specified; see below.

% ==============================================================================
\subsection{Load class}
% ==============================================================================

To load the class use
\begin{verbatim}
  \documentclass{goose-thesis}
\end{verbatim}
%
To use customized fonts, the documents has to be compiled using XeLaTeX. For example:
\begin{verbatim}
  %!TEX program = XeLaTeX
  \documentclass[garamond]{goose-thesis}
\end{verbatim}
%
The following fonts are available:
%
\begin{itemize}
  %
  \item \texttt{garamond}
  \item \texttt{times}
  \item \texttt{verdana}
  %
\end{itemize}
%
%
Furthermore the following options are available
%
\begin{itemize}
  %
  \item \texttt{narrow}: widen the margins of the page, useful during the review process;
  \item \texttt{doublespacing}: set the line-spacing to double, useful during the review process.
  \item \texttt{namecite}: use names instead of number of citations.
  \item \texttt{sectionbib}: include the bibliography at the end of each chapter.
  %
\end{itemize}
%

% ==============================================================================
\subsection{Title and author}
% ==============================================================================

%
\begin{itemize}
%
\item The \textit{title} is specified using
\begin{verbatim}
  \title{...}
\end{verbatim}
%
\item The \textit{author} is specified using
\begin{verbatim}
  \author{...}
\end{verbatim}
%
\item Additionally one could decide to change the author of the PDF-document
\begin{verbatim}
  \hypersetup{pdfauthor={...}}
\end{verbatim}
%
\end{itemize}
%

% ==============================================================================
\section{Main text}
% ==============================================================================

In the simplest form the thesis will have one bibliography at the end of the documents. It also possible to include a bibliography at the end of each chapter, see Section~\ref{sec:sectionbib}.

% ==============================================================================
\subsection{Structure}
% ==============================================================================

The thesis comprises of a main \TeX-files and \TeX-files for each chapter. Furthermore a \texttt{Makefile} can be used to gather the compilation instructions. The suggested structure is as follows
%
\begin{verbatim}
  main.tex
  example_chapter1.tex
  example_chapter2.tex
  ...
  library.bib
  figures/
\end{verbatim}
%

% ==============================================================================
\subsection{Main document}
% ==============================================================================

The layout is as follows
\begin{mdframed}
\begin{verbatim}
\documentclass[options]{goose-thesis}

\title{...}

\author{...}

...

\begin{document}

  \maketitle

  \setcounter{tocdepth}{0}
  \tableofcontents

  \cleardoublepage
  \include{example_chapter1}

  \cleardoublepage
  \include{example_chapter2}

  ...

  \bibliography{...}

\end{document}
\end{verbatim}
\end{mdframed}

Herein the chapters have been included as separate files. Notice that there is a single \verb|\bibliography{...}| entry at the end of the main document. None of the chapters (in the \verb|\include{...}| command) will have such an entry.

% ==============================================================================
\subsection{Chapters}
% ==============================================================================

The layout is as follows
\begin{mdframed}
\begin{verbatim}
\begin{chapter}

\begin{frontmatter}

\begin{abstract}
...
\end{abstract}

\keywords{...}

\begin{remark}
...
\end{remark}

\end{frontmatter}

...



\appendix

...

\end{verbatim}
\end{mdframed}

Notice how each of the chapters has its own appendix.

% ==============================================================================
\subsection{Compilation}
% ==============================================================================

The compilation can be done by compiling the main-file just like any other file.

% ==============================================================================
\section{Main text -- multiple bibliographies}
\label{sec:sectionbib}
% ==============================================================================

% ==============================================================================
\subsection{Structure}
% ==============================================================================

The thesis comprises of a main \TeX-files and \TeX-files for each chapter. Furthermore a \texttt{Makefile} can be used to gather the compilation instructions. The suggested structure is as follows
%
\begin{verbatim}
  main.tex
  example_chapter1.tex
  example_chapter2.tex
  ...
  library.bib
  figures/
\end{verbatim}
%

% ==============================================================================
\subsection{Main document}
% ==============================================================================

The layout is as follows
\begin{mdframed}
\begin{verbatim}
\documentclass[sectionbib,...]{goose-thesis}

\title{...}

\author{...}

...

\begin{document}

  \maketitle

  \setcounter{tocdepth}{0}
  \tableofcontents

  \cleardoublepage
  \include{example_chapter1}

  \cleardoublepage
  \include{example_chapter2}

  ...

\end{document}
\end{verbatim}
\end{mdframed}

Notice that in this case the main file does not have a \verb|\bibliography{...}| command. Rather, each of the chapter contains this command (unless of course that are no citations in that chapter). Also notice that the \verb|sectionbib| option has been used.

% ==============================================================================
\subsection{Chapters}
% ==============================================================================

The layout is as follows
\begin{mdframed}
\begin{verbatim}
\begin{chapter}

\begin{frontmatter}

\begin{abstract}
...
\end{abstract}

\keywords{...}

\begin{remark}
...
\end{remark}

\end{frontmatter}

...

\bibliography{...}

\appendix

...

\end{verbatim}
\end{mdframed}

Notice the \verb|\bibliography{...}| command at the end of the chapter.

% ==============================================================================
\subsection{Compilation}
% ==============================================================================

In this case the compilation is a bit more involved, as several bibliographies have to be created. For this example the steps are included in the following \texttt{Makefile}:

\begin{mdframed}
\begin{verbatim}
all:
  xelatex -interaction=nonstopmode example.tex
  xelatex -interaction=nonstopmode example.tex
  bibtex example_chapter1
  bibtex example_chapter2
  xelatex -interaction=nonstopmode example.tex
  xelatex -interaction=nonstopmode example.tex

clean:
  rm *.aux *.bbl *.log *.out *.pdf *.toc *.blg *.fls *.fdb_latexmk
\end{verbatim}
\end{mdframed}

% ==============================================================================
\section{Citations}
% ==============================================================================

Citations and references are handled using \texttt{natbib}. To cite use
\begin{verbatim}
  \citep{...}  (or \cite{...})
  \citet{...}
\end{verbatim}
The former only inserted a citation as number. For example \citep{geus}. The latter also includes the name(s) of the author(s). For example \citet{geus,wiki}.

The bibliography information is stored in a \texttt{bib}-file, which is included using
\begin{verbatim}
  \bibliography{...}
\end{verbatim}
This command creates a chapter section ``References'' with the bibliography. By default number citations are used, in which the references appear in the order in which they were cited. In the case that the \verb|namecite| option is used, the citations appear in names in alphabetical order.

Note that a large part of the formatting of Bib\TeX~depends on the formatting of the \texttt{bib}-file. A Python-script \texttt{bibparse} is available to automatically clean-up the formatting of the \texttt{bib}-file. An updated \texttt{unsrtnat.bst} is available that includes the \texttt{eprint} field.

% ==============================================================================
\bibliography{example_refs}
% ==============================================================================

\appendix

% ==============================================================================
\section{Some appendix}
% ==============================================================================


  \cleardoublepage
  
\chapter{Another chapter}

\begin{frontmatter}

\begin{abstract}
This is another chapter.
\end{abstract}

\end{frontmatter}

\section{With a section}

And a citation \citep{wiki}.

\bibliography{example_refs}

\appendix

\section{And an appendix}


  ...

\end{document}
\end{verbatim}
\end{mdframed}

Notice that in this case the main file does not have a \verb|\bibliography{...}| command. Rather, each of the chapter contains this command (unless of course that are no citations in that chapter). Also notice that the \verb|sectionbib| option has been used.

% ==============================================================================
\subsection{Chapters}
% ==============================================================================

The layout is as follows
\begin{mdframed}
\begin{verbatim}
\begin{chapter}

\begin{frontmatter}

\begin{abstract}
...
\end{abstract}

\keywords{...}

\begin{remark}
...
\end{remark}

\end{frontmatter}

...

\bibliography{...}

\appendix

...

\end{verbatim}
\end{mdframed}

Notice the \verb|\bibliography{...}| command at the end of the chapter.

% ==============================================================================
\subsection{Compilation}
% ==============================================================================

In this case the compilation is a bit more involved, as several bibliographies have to be created. For this example the steps are included in the following \texttt{Makefile}:

\begin{mdframed}
\begin{verbatim}
all:
  xelatex -interaction=nonstopmode example.tex
  xelatex -interaction=nonstopmode example.tex
  bibtex example_chapter1
  bibtex example_chapter2
  xelatex -interaction=nonstopmode example.tex
  xelatex -interaction=nonstopmode example.tex

clean:
  rm *.aux *.bbl *.log *.out *.pdf *.toc *.blg *.fls *.fdb_latexmk
\end{verbatim}
\end{mdframed}

% ==============================================================================
\section{Citations}
% ==============================================================================

Citations and references are handled using \texttt{natbib}. To cite use
\begin{verbatim}
  \citep{...}  (or \cite{...})
  \citet{...}
\end{verbatim}
The former only inserted a citation as number. For example \citep{geus}. The latter also includes the name(s) of the author(s). For example \citet{geus,wiki}.

The bibliography information is stored in a \texttt{bib}-file, which is included using
\begin{verbatim}
  \bibliography{...}
\end{verbatim}
This command creates a chapter section ``References'' with the bibliography. By default number citations are used, in which the references appear in the order in which they were cited. In the case that the \verb|namecite| option is used, the citations appear in names in alphabetical order.

Note that a large part of the formatting of Bib\TeX~depends on the formatting of the \texttt{bib}-file. A Python-script \texttt{bibparse} is available to automatically clean-up the formatting of the \texttt{bib}-file. An updated \texttt{unsrtnat.bst} is available that includes the \texttt{eprint} field.

% ==============================================================================
\bibliography{example_refs}
% ==============================================================================

\appendix

% ==============================================================================
\section{Some appendix}
% ==============================================================================


  \cleardoublepage
  
\chapter{Another chapter}

\begin{frontmatter}

\begin{abstract}
This is another chapter.
\end{abstract}

\end{frontmatter}

\section{With a section}

And a citation \citep{wiki}.

\bibliography{example_refs}

\appendix

\section{And an appendix}


  ...

\end{document}
\end{verbatim}
\end{mdframed}

Notice that in this case the main file does not have a \verb|\bibliography{...}| command. Rather, each of the chapter contains this command (unless of course that are no citations in that chapter). Also notice that the \verb|sectionbib| option has been used.

% ==============================================================================
\subsection{Chapters}
% ==============================================================================

The layout is as follows
\begin{mdframed}
\begin{verbatim}
\begin{chapter}

\begin{frontmatter}

\begin{abstract}
...
\end{abstract}

\keywords{...}

\begin{remark}
...
\end{remark}

\end{frontmatter}

...

\bibliography{...}

\appendix

...

\end{verbatim}
\end{mdframed}

Notice the \verb|\bibliography{...}| command at the end of the chapter.

% ==============================================================================
\subsection{Compilation}
% ==============================================================================

In this case the compilation is a bit more involved, as several bibliographies have to be created. For this example the steps are included in the following \texttt{Makefile}:

\begin{mdframed}
\begin{verbatim}
all:
  xelatex -interaction=nonstopmode example.tex
  xelatex -interaction=nonstopmode example.tex
  bibtex example_chapter1
  bibtex example_chapter2
  xelatex -interaction=nonstopmode example.tex
  xelatex -interaction=nonstopmode example.tex

clean:
  rm *.aux *.bbl *.log *.out *.pdf *.toc *.blg *.fls *.fdb_latexmk
\end{verbatim}
\end{mdframed}

% ==============================================================================
\section{Citations}
% ==============================================================================

Citations and references are handled using \texttt{natbib}. To cite use
\begin{verbatim}
  \citep{...}  (or \cite{...})
  \citet{...}
\end{verbatim}
The former only inserted a citation as number. For example \citep{geus}. The latter also includes the name(s) of the author(s). For example \citet{geus,wiki}.

The bibliography information is stored in a \texttt{bib}-file, which is included using
\begin{verbatim}
  \bibliography{...}
\end{verbatim}
This command creates a chapter section ``References'' with the bibliography. By default number citations are used, in which the references appear in the order in which they were cited. In the case that the \verb|namecite| option is used, the citations appear in names in alphabetical order.

Note that a large part of the formatting of Bib\TeX~depends on the formatting of the \texttt{bib}-file. A Python-script \texttt{bibparse} is available to automatically clean-up the formatting of the \texttt{bib}-file. An updated \texttt{unsrtnat.bst} is available that includes the \texttt{eprint} field.

% ==============================================================================
\bibliography{example_refs}
% ==============================================================================

\appendix

% ==============================================================================
\section{Some appendix}
% ==============================================================================


  \cleardoublepage
  
\chapter{Another chapter}

\begin{frontmatter}

\begin{abstract}
This is another chapter.
\end{abstract}

\end{frontmatter}

\section{With a section}

And a citation \citep{wiki}.

\bibliography{example_refs}

\appendix

\section{And an appendix}


  ...

  \bibliography{...}

\end{document}
\end{verbatim}
\end{mdframed}

Herein the chapters have been included as separate files. Notice that there is a single \verb|\bibliography{...}| entry at the end of the main document. None of the chapters (in the \verb|\include{...}| command) will have such an entry.

% ==============================================================================
\subsection{Chapters}
% ==============================================================================

The layout is as follows
\begin{mdframed}
\begin{verbatim}
\begin{chapter}

\begin{frontmatter}

\begin{abstract}
...
\end{abstract}

\keywords{...}

\begin{remark}
...
\end{remark}

\end{frontmatter}

...



\appendix

...

\end{verbatim}
\end{mdframed}

Notice how each of the chapters has its own appendix.

% ==============================================================================
\subsection{Compilation}
% ==============================================================================

The compilation can be done by compiling the main-file just like any other file.

% ==============================================================================
\section{Main text -- multiple bibliographies}
\label{sec:sectionbib}
% ==============================================================================

% ==============================================================================
\subsection{Structure}
% ==============================================================================

The thesis comprises of a main \TeX-files and \TeX-files for each chapter. Furthermore a \texttt{Makefile} can be used to gather the compilation instructions. The suggested structure is as follows
%
\begin{verbatim}
  main.tex
  example_chapter1.tex
  example_chapter2.tex
  ...
  library.bib
  figures/
\end{verbatim}
%

% ==============================================================================
\subsection{Main document}
% ==============================================================================

The layout is as follows
\begin{mdframed}
\begin{verbatim}
\documentclass[sectionbib,...]{goose-thesis}

\title{...}

\author{...}

...

\begin{document}

  \maketitle

  \setcounter{tocdepth}{0}
  \tableofcontents

  \cleardoublepage
  
% ==============================================================================
\chapter{Document layout}
% ==============================================================================

\begin{frontmatter}

\begin{abstract}
This chapter describes the document layout, including the compilation instructions.
\end{abstract}

\keywords{\LaTeX; class; article}

\begin{remark}
  T.W.J. de Geus
\end{remark}

\end{frontmatter}

% ==============================================================================
\section{Preamble}
% ==============================================================================

% ==============================================================================
\subsection{Introduction}
% ==============================================================================

By default most of the standard \LaTeX-packages are loaded. Any of these packages can be re-loaded, with other defaults, without problems. In addition the title and author can be specified; see below.

% ==============================================================================
\subsection{Load class}
% ==============================================================================

To load the class use
\begin{verbatim}
  \documentclass{goose-thesis}
\end{verbatim}
%
To use customized fonts, the documents has to be compiled using XeLaTeX. For example:
\begin{verbatim}
  %!TEX program = XeLaTeX
  \documentclass[garamond]{goose-thesis}
\end{verbatim}
%
The following fonts are available:
%
\begin{itemize}
  %
  \item \texttt{garamond}
  \item \texttt{times}
  \item \texttt{verdana}
  %
\end{itemize}
%
%
Furthermore the following options are available
%
\begin{itemize}
  %
  \item \texttt{narrow}: widen the margins of the page, useful during the review process;
  \item \texttt{doublespacing}: set the line-spacing to double, useful during the review process.
  \item \texttt{namecite}: use names instead of number of citations.
  \item \texttt{sectionbib}: include the bibliography at the end of each chapter.
  %
\end{itemize}
%

% ==============================================================================
\subsection{Title and author}
% ==============================================================================

%
\begin{itemize}
%
\item The \textit{title} is specified using
\begin{verbatim}
  \title{...}
\end{verbatim}
%
\item The \textit{author} is specified using
\begin{verbatim}
  \author{...}
\end{verbatim}
%
\item Additionally one could decide to change the author of the PDF-document
\begin{verbatim}
  \hypersetup{pdfauthor={...}}
\end{verbatim}
%
\end{itemize}
%

% ==============================================================================
\section{Main text}
% ==============================================================================

In the simplest form the thesis will have one bibliography at the end of the documents. It also possible to include a bibliography at the end of each chapter, see Section~\ref{sec:sectionbib}.

% ==============================================================================
\subsection{Structure}
% ==============================================================================

The thesis comprises of a main \TeX-files and \TeX-files for each chapter. Furthermore a \texttt{Makefile} can be used to gather the compilation instructions. The suggested structure is as follows
%
\begin{verbatim}
  main.tex
  example_chapter1.tex
  example_chapter2.tex
  ...
  library.bib
  figures/
\end{verbatim}
%

% ==============================================================================
\subsection{Main document}
% ==============================================================================

The layout is as follows
\begin{mdframed}
\begin{verbatim}
\documentclass[options]{goose-thesis}

\title{...}

\author{...}

...

\begin{document}

  \maketitle

  \setcounter{tocdepth}{0}
  \tableofcontents

  \cleardoublepage
  
% ==============================================================================
\chapter{Document layout}
% ==============================================================================

\begin{frontmatter}

\begin{abstract}
This chapter describes the document layout, including the compilation instructions.
\end{abstract}

\keywords{\LaTeX; class; article}

\begin{remark}
  T.W.J. de Geus
\end{remark}

\end{frontmatter}

% ==============================================================================
\section{Preamble}
% ==============================================================================

% ==============================================================================
\subsection{Introduction}
% ==============================================================================

By default most of the standard \LaTeX-packages are loaded. Any of these packages can be re-loaded, with other defaults, without problems. In addition the title and author can be specified; see below.

% ==============================================================================
\subsection{Load class}
% ==============================================================================

To load the class use
\begin{verbatim}
  \documentclass{goose-thesis}
\end{verbatim}
%
To use customized fonts, the documents has to be compiled using XeLaTeX. For example:
\begin{verbatim}
  %!TEX program = XeLaTeX
  \documentclass[garamond]{goose-thesis}
\end{verbatim}
%
The following fonts are available:
%
\begin{itemize}
  %
  \item \texttt{garamond}
  \item \texttt{times}
  \item \texttt{verdana}
  %
\end{itemize}
%
%
Furthermore the following options are available
%
\begin{itemize}
  %
  \item \texttt{narrow}: widen the margins of the page, useful during the review process;
  \item \texttt{doublespacing}: set the line-spacing to double, useful during the review process.
  \item \texttt{namecite}: use names instead of number of citations.
  \item \texttt{sectionbib}: include the bibliography at the end of each chapter.
  %
\end{itemize}
%

% ==============================================================================
\subsection{Title and author}
% ==============================================================================

%
\begin{itemize}
%
\item The \textit{title} is specified using
\begin{verbatim}
  \title{...}
\end{verbatim}
%
\item The \textit{author} is specified using
\begin{verbatim}
  \author{...}
\end{verbatim}
%
\item Additionally one could decide to change the author of the PDF-document
\begin{verbatim}
  \hypersetup{pdfauthor={...}}
\end{verbatim}
%
\end{itemize}
%

% ==============================================================================
\section{Main text}
% ==============================================================================

In the simplest form the thesis will have one bibliography at the end of the documents. It also possible to include a bibliography at the end of each chapter, see Section~\ref{sec:sectionbib}.

% ==============================================================================
\subsection{Structure}
% ==============================================================================

The thesis comprises of a main \TeX-files and \TeX-files for each chapter. Furthermore a \texttt{Makefile} can be used to gather the compilation instructions. The suggested structure is as follows
%
\begin{verbatim}
  main.tex
  example_chapter1.tex
  example_chapter2.tex
  ...
  library.bib
  figures/
\end{verbatim}
%

% ==============================================================================
\subsection{Main document}
% ==============================================================================

The layout is as follows
\begin{mdframed}
\begin{verbatim}
\documentclass[options]{goose-thesis}

\title{...}

\author{...}

...

\begin{document}

  \maketitle

  \setcounter{tocdepth}{0}
  \tableofcontents

  \cleardoublepage
  
% ==============================================================================
\chapter{Document layout}
% ==============================================================================

\begin{frontmatter}

\begin{abstract}
This chapter describes the document layout, including the compilation instructions.
\end{abstract}

\keywords{\LaTeX; class; article}

\begin{remark}
  T.W.J. de Geus
\end{remark}

\end{frontmatter}

% ==============================================================================
\section{Preamble}
% ==============================================================================

% ==============================================================================
\subsection{Introduction}
% ==============================================================================

By default most of the standard \LaTeX-packages are loaded. Any of these packages can be re-loaded, with other defaults, without problems. In addition the title and author can be specified; see below.

% ==============================================================================
\subsection{Load class}
% ==============================================================================

To load the class use
\begin{verbatim}
  \documentclass{goose-thesis}
\end{verbatim}
%
To use customized fonts, the documents has to be compiled using XeLaTeX. For example:
\begin{verbatim}
  %!TEX program = XeLaTeX
  \documentclass[garamond]{goose-thesis}
\end{verbatim}
%
The following fonts are available:
%
\begin{itemize}
  %
  \item \texttt{garamond}
  \item \texttt{times}
  \item \texttt{verdana}
  %
\end{itemize}
%
%
Furthermore the following options are available
%
\begin{itemize}
  %
  \item \texttt{narrow}: widen the margins of the page, useful during the review process;
  \item \texttt{doublespacing}: set the line-spacing to double, useful during the review process.
  \item \texttt{namecite}: use names instead of number of citations.
  \item \texttt{sectionbib}: include the bibliography at the end of each chapter.
  %
\end{itemize}
%

% ==============================================================================
\subsection{Title and author}
% ==============================================================================

%
\begin{itemize}
%
\item The \textit{title} is specified using
\begin{verbatim}
  \title{...}
\end{verbatim}
%
\item The \textit{author} is specified using
\begin{verbatim}
  \author{...}
\end{verbatim}
%
\item Additionally one could decide to change the author of the PDF-document
\begin{verbatim}
  \hypersetup{pdfauthor={...}}
\end{verbatim}
%
\end{itemize}
%

% ==============================================================================
\section{Main text}
% ==============================================================================

In the simplest form the thesis will have one bibliography at the end of the documents. It also possible to include a bibliography at the end of each chapter, see Section~\ref{sec:sectionbib}.

% ==============================================================================
\subsection{Structure}
% ==============================================================================

The thesis comprises of a main \TeX-files and \TeX-files for each chapter. Furthermore a \texttt{Makefile} can be used to gather the compilation instructions. The suggested structure is as follows
%
\begin{verbatim}
  main.tex
  example_chapter1.tex
  example_chapter2.tex
  ...
  library.bib
  figures/
\end{verbatim}
%

% ==============================================================================
\subsection{Main document}
% ==============================================================================

The layout is as follows
\begin{mdframed}
\begin{verbatim}
\documentclass[options]{goose-thesis}

\title{...}

\author{...}

...

\begin{document}

  \maketitle

  \setcounter{tocdepth}{0}
  \tableofcontents

  \cleardoublepage
  \include{example_chapter1}

  \cleardoublepage
  \include{example_chapter2}

  ...

  \bibliography{...}

\end{document}
\end{verbatim}
\end{mdframed}

Herein the chapters have been included as separate files. Notice that there is a single \verb|\bibliography{...}| entry at the end of the main document. None of the chapters (in the \verb|\include{...}| command) will have such an entry.

% ==============================================================================
\subsection{Chapters}
% ==============================================================================

The layout is as follows
\begin{mdframed}
\begin{verbatim}
\begin{chapter}

\begin{frontmatter}

\begin{abstract}
...
\end{abstract}

\keywords{...}

\begin{remark}
...
\end{remark}

\end{frontmatter}

...



\appendix

...

\end{verbatim}
\end{mdframed}

Notice how each of the chapters has its own appendix.

% ==============================================================================
\subsection{Compilation}
% ==============================================================================

The compilation can be done by compiling the main-file just like any other file.

% ==============================================================================
\section{Main text -- multiple bibliographies}
\label{sec:sectionbib}
% ==============================================================================

% ==============================================================================
\subsection{Structure}
% ==============================================================================

The thesis comprises of a main \TeX-files and \TeX-files for each chapter. Furthermore a \texttt{Makefile} can be used to gather the compilation instructions. The suggested structure is as follows
%
\begin{verbatim}
  main.tex
  example_chapter1.tex
  example_chapter2.tex
  ...
  library.bib
  figures/
\end{verbatim}
%

% ==============================================================================
\subsection{Main document}
% ==============================================================================

The layout is as follows
\begin{mdframed}
\begin{verbatim}
\documentclass[sectionbib,...]{goose-thesis}

\title{...}

\author{...}

...

\begin{document}

  \maketitle

  \setcounter{tocdepth}{0}
  \tableofcontents

  \cleardoublepage
  \include{example_chapter1}

  \cleardoublepage
  \include{example_chapter2}

  ...

\end{document}
\end{verbatim}
\end{mdframed}

Notice that in this case the main file does not have a \verb|\bibliography{...}| command. Rather, each of the chapter contains this command (unless of course that are no citations in that chapter). Also notice that the \verb|sectionbib| option has been used.

% ==============================================================================
\subsection{Chapters}
% ==============================================================================

The layout is as follows
\begin{mdframed}
\begin{verbatim}
\begin{chapter}

\begin{frontmatter}

\begin{abstract}
...
\end{abstract}

\keywords{...}

\begin{remark}
...
\end{remark}

\end{frontmatter}

...

\bibliography{...}

\appendix

...

\end{verbatim}
\end{mdframed}

Notice the \verb|\bibliography{...}| command at the end of the chapter.

% ==============================================================================
\subsection{Compilation}
% ==============================================================================

In this case the compilation is a bit more involved, as several bibliographies have to be created. For this example the steps are included in the following \texttt{Makefile}:

\begin{mdframed}
\begin{verbatim}
all:
  xelatex -interaction=nonstopmode example.tex
  xelatex -interaction=nonstopmode example.tex
  bibtex example_chapter1
  bibtex example_chapter2
  xelatex -interaction=nonstopmode example.tex
  xelatex -interaction=nonstopmode example.tex

clean:
  rm *.aux *.bbl *.log *.out *.pdf *.toc *.blg *.fls *.fdb_latexmk
\end{verbatim}
\end{mdframed}

% ==============================================================================
\section{Citations}
% ==============================================================================

Citations and references are handled using \texttt{natbib}. To cite use
\begin{verbatim}
  \citep{...}  (or \cite{...})
  \citet{...}
\end{verbatim}
The former only inserted a citation as number. For example \citep{geus}. The latter also includes the name(s) of the author(s). For example \citet{geus,wiki}.

The bibliography information is stored in a \texttt{bib}-file, which is included using
\begin{verbatim}
  \bibliography{...}
\end{verbatim}
This command creates a chapter section ``References'' with the bibliography. By default number citations are used, in which the references appear in the order in which they were cited. In the case that the \verb|namecite| option is used, the citations appear in names in alphabetical order.

Note that a large part of the formatting of Bib\TeX~depends on the formatting of the \texttt{bib}-file. A Python-script \texttt{bibparse} is available to automatically clean-up the formatting of the \texttt{bib}-file. An updated \texttt{unsrtnat.bst} is available that includes the \texttt{eprint} field.

% ==============================================================================
\bibliography{example_refs}
% ==============================================================================

\appendix

% ==============================================================================
\section{Some appendix}
% ==============================================================================


  \cleardoublepage
  
\chapter{Another chapter}

\begin{frontmatter}

\begin{abstract}
This is another chapter.
\end{abstract}

\end{frontmatter}

\section{With a section}

And a citation \citep{wiki}.

\bibliography{example_refs}

\appendix

\section{And an appendix}


  ...

  \bibliography{...}

\end{document}
\end{verbatim}
\end{mdframed}

Herein the chapters have been included as separate files. Notice that there is a single \verb|\bibliography{...}| entry at the end of the main document. None of the chapters (in the \verb|\include{...}| command) will have such an entry.

% ==============================================================================
\subsection{Chapters}
% ==============================================================================

The layout is as follows
\begin{mdframed}
\begin{verbatim}
\begin{chapter}

\begin{frontmatter}

\begin{abstract}
...
\end{abstract}

\keywords{...}

\begin{remark}
...
\end{remark}

\end{frontmatter}

...



\appendix

...

\end{verbatim}
\end{mdframed}

Notice how each of the chapters has its own appendix.

% ==============================================================================
\subsection{Compilation}
% ==============================================================================

The compilation can be done by compiling the main-file just like any other file.

% ==============================================================================
\section{Main text -- multiple bibliographies}
\label{sec:sectionbib}
% ==============================================================================

% ==============================================================================
\subsection{Structure}
% ==============================================================================

The thesis comprises of a main \TeX-files and \TeX-files for each chapter. Furthermore a \texttt{Makefile} can be used to gather the compilation instructions. The suggested structure is as follows
%
\begin{verbatim}
  main.tex
  example_chapter1.tex
  example_chapter2.tex
  ...
  library.bib
  figures/
\end{verbatim}
%

% ==============================================================================
\subsection{Main document}
% ==============================================================================

The layout is as follows
\begin{mdframed}
\begin{verbatim}
\documentclass[sectionbib,...]{goose-thesis}

\title{...}

\author{...}

...

\begin{document}

  \maketitle

  \setcounter{tocdepth}{0}
  \tableofcontents

  \cleardoublepage
  
% ==============================================================================
\chapter{Document layout}
% ==============================================================================

\begin{frontmatter}

\begin{abstract}
This chapter describes the document layout, including the compilation instructions.
\end{abstract}

\keywords{\LaTeX; class; article}

\begin{remark}
  T.W.J. de Geus
\end{remark}

\end{frontmatter}

% ==============================================================================
\section{Preamble}
% ==============================================================================

% ==============================================================================
\subsection{Introduction}
% ==============================================================================

By default most of the standard \LaTeX-packages are loaded. Any of these packages can be re-loaded, with other defaults, without problems. In addition the title and author can be specified; see below.

% ==============================================================================
\subsection{Load class}
% ==============================================================================

To load the class use
\begin{verbatim}
  \documentclass{goose-thesis}
\end{verbatim}
%
To use customized fonts, the documents has to be compiled using XeLaTeX. For example:
\begin{verbatim}
  %!TEX program = XeLaTeX
  \documentclass[garamond]{goose-thesis}
\end{verbatim}
%
The following fonts are available:
%
\begin{itemize}
  %
  \item \texttt{garamond}
  \item \texttt{times}
  \item \texttt{verdana}
  %
\end{itemize}
%
%
Furthermore the following options are available
%
\begin{itemize}
  %
  \item \texttt{narrow}: widen the margins of the page, useful during the review process;
  \item \texttt{doublespacing}: set the line-spacing to double, useful during the review process.
  \item \texttt{namecite}: use names instead of number of citations.
  \item \texttt{sectionbib}: include the bibliography at the end of each chapter.
  %
\end{itemize}
%

% ==============================================================================
\subsection{Title and author}
% ==============================================================================

%
\begin{itemize}
%
\item The \textit{title} is specified using
\begin{verbatim}
  \title{...}
\end{verbatim}
%
\item The \textit{author} is specified using
\begin{verbatim}
  \author{...}
\end{verbatim}
%
\item Additionally one could decide to change the author of the PDF-document
\begin{verbatim}
  \hypersetup{pdfauthor={...}}
\end{verbatim}
%
\end{itemize}
%

% ==============================================================================
\section{Main text}
% ==============================================================================

In the simplest form the thesis will have one bibliography at the end of the documents. It also possible to include a bibliography at the end of each chapter, see Section~\ref{sec:sectionbib}.

% ==============================================================================
\subsection{Structure}
% ==============================================================================

The thesis comprises of a main \TeX-files and \TeX-files for each chapter. Furthermore a \texttt{Makefile} can be used to gather the compilation instructions. The suggested structure is as follows
%
\begin{verbatim}
  main.tex
  example_chapter1.tex
  example_chapter2.tex
  ...
  library.bib
  figures/
\end{verbatim}
%

% ==============================================================================
\subsection{Main document}
% ==============================================================================

The layout is as follows
\begin{mdframed}
\begin{verbatim}
\documentclass[options]{goose-thesis}

\title{...}

\author{...}

...

\begin{document}

  \maketitle

  \setcounter{tocdepth}{0}
  \tableofcontents

  \cleardoublepage
  \include{example_chapter1}

  \cleardoublepage
  \include{example_chapter2}

  ...

  \bibliography{...}

\end{document}
\end{verbatim}
\end{mdframed}

Herein the chapters have been included as separate files. Notice that there is a single \verb|\bibliography{...}| entry at the end of the main document. None of the chapters (in the \verb|\include{...}| command) will have such an entry.

% ==============================================================================
\subsection{Chapters}
% ==============================================================================

The layout is as follows
\begin{mdframed}
\begin{verbatim}
\begin{chapter}

\begin{frontmatter}

\begin{abstract}
...
\end{abstract}

\keywords{...}

\begin{remark}
...
\end{remark}

\end{frontmatter}

...



\appendix

...

\end{verbatim}
\end{mdframed}

Notice how each of the chapters has its own appendix.

% ==============================================================================
\subsection{Compilation}
% ==============================================================================

The compilation can be done by compiling the main-file just like any other file.

% ==============================================================================
\section{Main text -- multiple bibliographies}
\label{sec:sectionbib}
% ==============================================================================

% ==============================================================================
\subsection{Structure}
% ==============================================================================

The thesis comprises of a main \TeX-files and \TeX-files for each chapter. Furthermore a \texttt{Makefile} can be used to gather the compilation instructions. The suggested structure is as follows
%
\begin{verbatim}
  main.tex
  example_chapter1.tex
  example_chapter2.tex
  ...
  library.bib
  figures/
\end{verbatim}
%

% ==============================================================================
\subsection{Main document}
% ==============================================================================

The layout is as follows
\begin{mdframed}
\begin{verbatim}
\documentclass[sectionbib,...]{goose-thesis}

\title{...}

\author{...}

...

\begin{document}

  \maketitle

  \setcounter{tocdepth}{0}
  \tableofcontents

  \cleardoublepage
  \include{example_chapter1}

  \cleardoublepage
  \include{example_chapter2}

  ...

\end{document}
\end{verbatim}
\end{mdframed}

Notice that in this case the main file does not have a \verb|\bibliography{...}| command. Rather, each of the chapter contains this command (unless of course that are no citations in that chapter). Also notice that the \verb|sectionbib| option has been used.

% ==============================================================================
\subsection{Chapters}
% ==============================================================================

The layout is as follows
\begin{mdframed}
\begin{verbatim}
\begin{chapter}

\begin{frontmatter}

\begin{abstract}
...
\end{abstract}

\keywords{...}

\begin{remark}
...
\end{remark}

\end{frontmatter}

...

\bibliography{...}

\appendix

...

\end{verbatim}
\end{mdframed}

Notice the \verb|\bibliography{...}| command at the end of the chapter.

% ==============================================================================
\subsection{Compilation}
% ==============================================================================

In this case the compilation is a bit more involved, as several bibliographies have to be created. For this example the steps are included in the following \texttt{Makefile}:

\begin{mdframed}
\begin{verbatim}
all:
  xelatex -interaction=nonstopmode example.tex
  xelatex -interaction=nonstopmode example.tex
  bibtex example_chapter1
  bibtex example_chapter2
  xelatex -interaction=nonstopmode example.tex
  xelatex -interaction=nonstopmode example.tex

clean:
  rm *.aux *.bbl *.log *.out *.pdf *.toc *.blg *.fls *.fdb_latexmk
\end{verbatim}
\end{mdframed}

% ==============================================================================
\section{Citations}
% ==============================================================================

Citations and references are handled using \texttt{natbib}. To cite use
\begin{verbatim}
  \citep{...}  (or \cite{...})
  \citet{...}
\end{verbatim}
The former only inserted a citation as number. For example \citep{geus}. The latter also includes the name(s) of the author(s). For example \citet{geus,wiki}.

The bibliography information is stored in a \texttt{bib}-file, which is included using
\begin{verbatim}
  \bibliography{...}
\end{verbatim}
This command creates a chapter section ``References'' with the bibliography. By default number citations are used, in which the references appear in the order in which they were cited. In the case that the \verb|namecite| option is used, the citations appear in names in alphabetical order.

Note that a large part of the formatting of Bib\TeX~depends on the formatting of the \texttt{bib}-file. A Python-script \texttt{bibparse} is available to automatically clean-up the formatting of the \texttt{bib}-file. An updated \texttt{unsrtnat.bst} is available that includes the \texttt{eprint} field.

% ==============================================================================
\bibliography{example_refs}
% ==============================================================================

\appendix

% ==============================================================================
\section{Some appendix}
% ==============================================================================


  \cleardoublepage
  
\chapter{Another chapter}

\begin{frontmatter}

\begin{abstract}
This is another chapter.
\end{abstract}

\end{frontmatter}

\section{With a section}

And a citation \citep{wiki}.

\bibliography{example_refs}

\appendix

\section{And an appendix}


  ...

\end{document}
\end{verbatim}
\end{mdframed}

Notice that in this case the main file does not have a \verb|\bibliography{...}| command. Rather, each of the chapter contains this command (unless of course that are no citations in that chapter). Also notice that the \verb|sectionbib| option has been used.

% ==============================================================================
\subsection{Chapters}
% ==============================================================================

The layout is as follows
\begin{mdframed}
\begin{verbatim}
\begin{chapter}

\begin{frontmatter}

\begin{abstract}
...
\end{abstract}

\keywords{...}

\begin{remark}
...
\end{remark}

\end{frontmatter}

...

\bibliography{...}

\appendix

...

\end{verbatim}
\end{mdframed}

Notice the \verb|\bibliography{...}| command at the end of the chapter.

% ==============================================================================
\subsection{Compilation}
% ==============================================================================

In this case the compilation is a bit more involved, as several bibliographies have to be created. For this example the steps are included in the following \texttt{Makefile}:

\begin{mdframed}
\begin{verbatim}
all:
  xelatex -interaction=nonstopmode example.tex
  xelatex -interaction=nonstopmode example.tex
  bibtex example_chapter1
  bibtex example_chapter2
  xelatex -interaction=nonstopmode example.tex
  xelatex -interaction=nonstopmode example.tex

clean:
  rm *.aux *.bbl *.log *.out *.pdf *.toc *.blg *.fls *.fdb_latexmk
\end{verbatim}
\end{mdframed}

% ==============================================================================
\section{Citations}
% ==============================================================================

Citations and references are handled using \texttt{natbib}. To cite use
\begin{verbatim}
  \citep{...}  (or \cite{...})
  \citet{...}
\end{verbatim}
The former only inserted a citation as number. For example \citep{geus}. The latter also includes the name(s) of the author(s). For example \citet{geus,wiki}.

The bibliography information is stored in a \texttt{bib}-file, which is included using
\begin{verbatim}
  \bibliography{...}
\end{verbatim}
This command creates a chapter section ``References'' with the bibliography. By default number citations are used, in which the references appear in the order in which they were cited. In the case that the \verb|namecite| option is used, the citations appear in names in alphabetical order.

Note that a large part of the formatting of Bib\TeX~depends on the formatting of the \texttt{bib}-file. A Python-script \texttt{bibparse} is available to automatically clean-up the formatting of the \texttt{bib}-file. An updated \texttt{unsrtnat.bst} is available that includes the \texttt{eprint} field.

% ==============================================================================
\bibliography{example_refs}
% ==============================================================================

\appendix

% ==============================================================================
\section{Some appendix}
% ==============================================================================


  \cleardoublepage
  
\chapter{Another chapter}

\begin{frontmatter}

\begin{abstract}
This is another chapter.
\end{abstract}

\end{frontmatter}

\section{With a section}

And a citation \citep{wiki}.

\bibliography{example_refs}

\appendix

\section{And an appendix}


  ...

  \bibliography{...}

\end{document}
\end{verbatim}
\end{mdframed}

Herein the chapters have been included as separate files. Notice that there is a single \verb|\bibliography{...}| entry at the end of the main document. None of the chapters (in the \verb|\include{...}| command) will have such an entry.

% ==============================================================================
\subsection{Chapters}
% ==============================================================================

The layout is as follows
\begin{mdframed}
\begin{verbatim}
\begin{chapter}

\begin{frontmatter}

\begin{abstract}
...
\end{abstract}

\keywords{...}

\begin{remark}
...
\end{remark}

\end{frontmatter}

...



\appendix

...

\end{verbatim}
\end{mdframed}

Notice how each of the chapters has its own appendix.

% ==============================================================================
\subsection{Compilation}
% ==============================================================================

The compilation can be done by compiling the main-file just like any other file.

% ==============================================================================
\section{Main text -- multiple bibliographies}
\label{sec:sectionbib}
% ==============================================================================

% ==============================================================================
\subsection{Structure}
% ==============================================================================

The thesis comprises of a main \TeX-files and \TeX-files for each chapter. Furthermore a \texttt{Makefile} can be used to gather the compilation instructions. The suggested structure is as follows
%
\begin{verbatim}
  main.tex
  example_chapter1.tex
  example_chapter2.tex
  ...
  library.bib
  figures/
\end{verbatim}
%

% ==============================================================================
\subsection{Main document}
% ==============================================================================

The layout is as follows
\begin{mdframed}
\begin{verbatim}
\documentclass[sectionbib,...]{goose-thesis}

\title{...}

\author{...}

...

\begin{document}

  \maketitle

  \setcounter{tocdepth}{0}
  \tableofcontents

  \cleardoublepage
  
% ==============================================================================
\chapter{Document layout}
% ==============================================================================

\begin{frontmatter}

\begin{abstract}
This chapter describes the document layout, including the compilation instructions.
\end{abstract}

\keywords{\LaTeX; class; article}

\begin{remark}
  T.W.J. de Geus
\end{remark}

\end{frontmatter}

% ==============================================================================
\section{Preamble}
% ==============================================================================

% ==============================================================================
\subsection{Introduction}
% ==============================================================================

By default most of the standard \LaTeX-packages are loaded. Any of these packages can be re-loaded, with other defaults, without problems. In addition the title and author can be specified; see below.

% ==============================================================================
\subsection{Load class}
% ==============================================================================

To load the class use
\begin{verbatim}
  \documentclass{goose-thesis}
\end{verbatim}
%
To use customized fonts, the documents has to be compiled using XeLaTeX. For example:
\begin{verbatim}
  %!TEX program = XeLaTeX
  \documentclass[garamond]{goose-thesis}
\end{verbatim}
%
The following fonts are available:
%
\begin{itemize}
  %
  \item \texttt{garamond}
  \item \texttt{times}
  \item \texttt{verdana}
  %
\end{itemize}
%
%
Furthermore the following options are available
%
\begin{itemize}
  %
  \item \texttt{narrow}: widen the margins of the page, useful during the review process;
  \item \texttt{doublespacing}: set the line-spacing to double, useful during the review process.
  \item \texttt{namecite}: use names instead of number of citations.
  \item \texttt{sectionbib}: include the bibliography at the end of each chapter.
  %
\end{itemize}
%

% ==============================================================================
\subsection{Title and author}
% ==============================================================================

%
\begin{itemize}
%
\item The \textit{title} is specified using
\begin{verbatim}
  \title{...}
\end{verbatim}
%
\item The \textit{author} is specified using
\begin{verbatim}
  \author{...}
\end{verbatim}
%
\item Additionally one could decide to change the author of the PDF-document
\begin{verbatim}
  \hypersetup{pdfauthor={...}}
\end{verbatim}
%
\end{itemize}
%

% ==============================================================================
\section{Main text}
% ==============================================================================

In the simplest form the thesis will have one bibliography at the end of the documents. It also possible to include a bibliography at the end of each chapter, see Section~\ref{sec:sectionbib}.

% ==============================================================================
\subsection{Structure}
% ==============================================================================

The thesis comprises of a main \TeX-files and \TeX-files for each chapter. Furthermore a \texttt{Makefile} can be used to gather the compilation instructions. The suggested structure is as follows
%
\begin{verbatim}
  main.tex
  example_chapter1.tex
  example_chapter2.tex
  ...
  library.bib
  figures/
\end{verbatim}
%

% ==============================================================================
\subsection{Main document}
% ==============================================================================

The layout is as follows
\begin{mdframed}
\begin{verbatim}
\documentclass[options]{goose-thesis}

\title{...}

\author{...}

...

\begin{document}

  \maketitle

  \setcounter{tocdepth}{0}
  \tableofcontents

  \cleardoublepage
  
% ==============================================================================
\chapter{Document layout}
% ==============================================================================

\begin{frontmatter}

\begin{abstract}
This chapter describes the document layout, including the compilation instructions.
\end{abstract}

\keywords{\LaTeX; class; article}

\begin{remark}
  T.W.J. de Geus
\end{remark}

\end{frontmatter}

% ==============================================================================
\section{Preamble}
% ==============================================================================

% ==============================================================================
\subsection{Introduction}
% ==============================================================================

By default most of the standard \LaTeX-packages are loaded. Any of these packages can be re-loaded, with other defaults, without problems. In addition the title and author can be specified; see below.

% ==============================================================================
\subsection{Load class}
% ==============================================================================

To load the class use
\begin{verbatim}
  \documentclass{goose-thesis}
\end{verbatim}
%
To use customized fonts, the documents has to be compiled using XeLaTeX. For example:
\begin{verbatim}
  %!TEX program = XeLaTeX
  \documentclass[garamond]{goose-thesis}
\end{verbatim}
%
The following fonts are available:
%
\begin{itemize}
  %
  \item \texttt{garamond}
  \item \texttt{times}
  \item \texttt{verdana}
  %
\end{itemize}
%
%
Furthermore the following options are available
%
\begin{itemize}
  %
  \item \texttt{narrow}: widen the margins of the page, useful during the review process;
  \item \texttt{doublespacing}: set the line-spacing to double, useful during the review process.
  \item \texttt{namecite}: use names instead of number of citations.
  \item \texttt{sectionbib}: include the bibliography at the end of each chapter.
  %
\end{itemize}
%

% ==============================================================================
\subsection{Title and author}
% ==============================================================================

%
\begin{itemize}
%
\item The \textit{title} is specified using
\begin{verbatim}
  \title{...}
\end{verbatim}
%
\item The \textit{author} is specified using
\begin{verbatim}
  \author{...}
\end{verbatim}
%
\item Additionally one could decide to change the author of the PDF-document
\begin{verbatim}
  \hypersetup{pdfauthor={...}}
\end{verbatim}
%
\end{itemize}
%

% ==============================================================================
\section{Main text}
% ==============================================================================

In the simplest form the thesis will have one bibliography at the end of the documents. It also possible to include a bibliography at the end of each chapter, see Section~\ref{sec:sectionbib}.

% ==============================================================================
\subsection{Structure}
% ==============================================================================

The thesis comprises of a main \TeX-files and \TeX-files for each chapter. Furthermore a \texttt{Makefile} can be used to gather the compilation instructions. The suggested structure is as follows
%
\begin{verbatim}
  main.tex
  example_chapter1.tex
  example_chapter2.tex
  ...
  library.bib
  figures/
\end{verbatim}
%

% ==============================================================================
\subsection{Main document}
% ==============================================================================

The layout is as follows
\begin{mdframed}
\begin{verbatim}
\documentclass[options]{goose-thesis}

\title{...}

\author{...}

...

\begin{document}

  \maketitle

  \setcounter{tocdepth}{0}
  \tableofcontents

  \cleardoublepage
  \include{example_chapter1}

  \cleardoublepage
  \include{example_chapter2}

  ...

  \bibliography{...}

\end{document}
\end{verbatim}
\end{mdframed}

Herein the chapters have been included as separate files. Notice that there is a single \verb|\bibliography{...}| entry at the end of the main document. None of the chapters (in the \verb|\include{...}| command) will have such an entry.

% ==============================================================================
\subsection{Chapters}
% ==============================================================================

The layout is as follows
\begin{mdframed}
\begin{verbatim}
\begin{chapter}

\begin{frontmatter}

\begin{abstract}
...
\end{abstract}

\keywords{...}

\begin{remark}
...
\end{remark}

\end{frontmatter}

...



\appendix

...

\end{verbatim}
\end{mdframed}

Notice how each of the chapters has its own appendix.

% ==============================================================================
\subsection{Compilation}
% ==============================================================================

The compilation can be done by compiling the main-file just like any other file.

% ==============================================================================
\section{Main text -- multiple bibliographies}
\label{sec:sectionbib}
% ==============================================================================

% ==============================================================================
\subsection{Structure}
% ==============================================================================

The thesis comprises of a main \TeX-files and \TeX-files for each chapter. Furthermore a \texttt{Makefile} can be used to gather the compilation instructions. The suggested structure is as follows
%
\begin{verbatim}
  main.tex
  example_chapter1.tex
  example_chapter2.tex
  ...
  library.bib
  figures/
\end{verbatim}
%

% ==============================================================================
\subsection{Main document}
% ==============================================================================

The layout is as follows
\begin{mdframed}
\begin{verbatim}
\documentclass[sectionbib,...]{goose-thesis}

\title{...}

\author{...}

...

\begin{document}

  \maketitle

  \setcounter{tocdepth}{0}
  \tableofcontents

  \cleardoublepage
  \include{example_chapter1}

  \cleardoublepage
  \include{example_chapter2}

  ...

\end{document}
\end{verbatim}
\end{mdframed}

Notice that in this case the main file does not have a \verb|\bibliography{...}| command. Rather, each of the chapter contains this command (unless of course that are no citations in that chapter). Also notice that the \verb|sectionbib| option has been used.

% ==============================================================================
\subsection{Chapters}
% ==============================================================================

The layout is as follows
\begin{mdframed}
\begin{verbatim}
\begin{chapter}

\begin{frontmatter}

\begin{abstract}
...
\end{abstract}

\keywords{...}

\begin{remark}
...
\end{remark}

\end{frontmatter}

...

\bibliography{...}

\appendix

...

\end{verbatim}
\end{mdframed}

Notice the \verb|\bibliography{...}| command at the end of the chapter.

% ==============================================================================
\subsection{Compilation}
% ==============================================================================

In this case the compilation is a bit more involved, as several bibliographies have to be created. For this example the steps are included in the following \texttt{Makefile}:

\begin{mdframed}
\begin{verbatim}
all:
  xelatex -interaction=nonstopmode example.tex
  xelatex -interaction=nonstopmode example.tex
  bibtex example_chapter1
  bibtex example_chapter2
  xelatex -interaction=nonstopmode example.tex
  xelatex -interaction=nonstopmode example.tex

clean:
  rm *.aux *.bbl *.log *.out *.pdf *.toc *.blg *.fls *.fdb_latexmk
\end{verbatim}
\end{mdframed}

% ==============================================================================
\section{Citations}
% ==============================================================================

Citations and references are handled using \texttt{natbib}. To cite use
\begin{verbatim}
  \citep{...}  (or \cite{...})
  \citet{...}
\end{verbatim}
The former only inserted a citation as number. For example \citep{geus}. The latter also includes the name(s) of the author(s). For example \citet{geus,wiki}.

The bibliography information is stored in a \texttt{bib}-file, which is included using
\begin{verbatim}
  \bibliography{...}
\end{verbatim}
This command creates a chapter section ``References'' with the bibliography. By default number citations are used, in which the references appear in the order in which they were cited. In the case that the \verb|namecite| option is used, the citations appear in names in alphabetical order.

Note that a large part of the formatting of Bib\TeX~depends on the formatting of the \texttt{bib}-file. A Python-script \texttt{bibparse} is available to automatically clean-up the formatting of the \texttt{bib}-file. An updated \texttt{unsrtnat.bst} is available that includes the \texttt{eprint} field.

% ==============================================================================
\bibliography{example_refs}
% ==============================================================================

\appendix

% ==============================================================================
\section{Some appendix}
% ==============================================================================


  \cleardoublepage
  
\chapter{Another chapter}

\begin{frontmatter}

\begin{abstract}
This is another chapter.
\end{abstract}

\end{frontmatter}

\section{With a section}

And a citation \citep{wiki}.

\bibliography{example_refs}

\appendix

\section{And an appendix}


  ...

  \bibliography{...}

\end{document}
\end{verbatim}
\end{mdframed}

Herein the chapters have been included as separate files. Notice that there is a single \verb|\bibliography{...}| entry at the end of the main document. None of the chapters (in the \verb|\include{...}| command) will have such an entry.

% ==============================================================================
\subsection{Chapters}
% ==============================================================================

The layout is as follows
\begin{mdframed}
\begin{verbatim}
\begin{chapter}

\begin{frontmatter}

\begin{abstract}
...
\end{abstract}

\keywords{...}

\begin{remark}
...
\end{remark}

\end{frontmatter}

...



\appendix

...

\end{verbatim}
\end{mdframed}

Notice how each of the chapters has its own appendix.

% ==============================================================================
\subsection{Compilation}
% ==============================================================================

The compilation can be done by compiling the main-file just like any other file.

% ==============================================================================
\section{Main text -- multiple bibliographies}
\label{sec:sectionbib}
% ==============================================================================

% ==============================================================================
\subsection{Structure}
% ==============================================================================

The thesis comprises of a main \TeX-files and \TeX-files for each chapter. Furthermore a \texttt{Makefile} can be used to gather the compilation instructions. The suggested structure is as follows
%
\begin{verbatim}
  main.tex
  example_chapter1.tex
  example_chapter2.tex
  ...
  library.bib
  figures/
\end{verbatim}
%

% ==============================================================================
\subsection{Main document}
% ==============================================================================

The layout is as follows
\begin{mdframed}
\begin{verbatim}
\documentclass[sectionbib,...]{goose-thesis}

\title{...}

\author{...}

...

\begin{document}

  \maketitle

  \setcounter{tocdepth}{0}
  \tableofcontents

  \cleardoublepage
  
% ==============================================================================
\chapter{Document layout}
% ==============================================================================

\begin{frontmatter}

\begin{abstract}
This chapter describes the document layout, including the compilation instructions.
\end{abstract}

\keywords{\LaTeX; class; article}

\begin{remark}
  T.W.J. de Geus
\end{remark}

\end{frontmatter}

% ==============================================================================
\section{Preamble}
% ==============================================================================

% ==============================================================================
\subsection{Introduction}
% ==============================================================================

By default most of the standard \LaTeX-packages are loaded. Any of these packages can be re-loaded, with other defaults, without problems. In addition the title and author can be specified; see below.

% ==============================================================================
\subsection{Load class}
% ==============================================================================

To load the class use
\begin{verbatim}
  \documentclass{goose-thesis}
\end{verbatim}
%
To use customized fonts, the documents has to be compiled using XeLaTeX. For example:
\begin{verbatim}
  %!TEX program = XeLaTeX
  \documentclass[garamond]{goose-thesis}
\end{verbatim}
%
The following fonts are available:
%
\begin{itemize}
  %
  \item \texttt{garamond}
  \item \texttt{times}
  \item \texttt{verdana}
  %
\end{itemize}
%
%
Furthermore the following options are available
%
\begin{itemize}
  %
  \item \texttt{narrow}: widen the margins of the page, useful during the review process;
  \item \texttt{doublespacing}: set the line-spacing to double, useful during the review process.
  \item \texttt{namecite}: use names instead of number of citations.
  \item \texttt{sectionbib}: include the bibliography at the end of each chapter.
  %
\end{itemize}
%

% ==============================================================================
\subsection{Title and author}
% ==============================================================================

%
\begin{itemize}
%
\item The \textit{title} is specified using
\begin{verbatim}
  \title{...}
\end{verbatim}
%
\item The \textit{author} is specified using
\begin{verbatim}
  \author{...}
\end{verbatim}
%
\item Additionally one could decide to change the author of the PDF-document
\begin{verbatim}
  \hypersetup{pdfauthor={...}}
\end{verbatim}
%
\end{itemize}
%

% ==============================================================================
\section{Main text}
% ==============================================================================

In the simplest form the thesis will have one bibliography at the end of the documents. It also possible to include a bibliography at the end of each chapter, see Section~\ref{sec:sectionbib}.

% ==============================================================================
\subsection{Structure}
% ==============================================================================

The thesis comprises of a main \TeX-files and \TeX-files for each chapter. Furthermore a \texttt{Makefile} can be used to gather the compilation instructions. The suggested structure is as follows
%
\begin{verbatim}
  main.tex
  example_chapter1.tex
  example_chapter2.tex
  ...
  library.bib
  figures/
\end{verbatim}
%

% ==============================================================================
\subsection{Main document}
% ==============================================================================

The layout is as follows
\begin{mdframed}
\begin{verbatim}
\documentclass[options]{goose-thesis}

\title{...}

\author{...}

...

\begin{document}

  \maketitle

  \setcounter{tocdepth}{0}
  \tableofcontents

  \cleardoublepage
  \include{example_chapter1}

  \cleardoublepage
  \include{example_chapter2}

  ...

  \bibliography{...}

\end{document}
\end{verbatim}
\end{mdframed}

Herein the chapters have been included as separate files. Notice that there is a single \verb|\bibliography{...}| entry at the end of the main document. None of the chapters (in the \verb|\include{...}| command) will have such an entry.

% ==============================================================================
\subsection{Chapters}
% ==============================================================================

The layout is as follows
\begin{mdframed}
\begin{verbatim}
\begin{chapter}

\begin{frontmatter}

\begin{abstract}
...
\end{abstract}

\keywords{...}

\begin{remark}
...
\end{remark}

\end{frontmatter}

...



\appendix

...

\end{verbatim}
\end{mdframed}

Notice how each of the chapters has its own appendix.

% ==============================================================================
\subsection{Compilation}
% ==============================================================================

The compilation can be done by compiling the main-file just like any other file.

% ==============================================================================
\section{Main text -- multiple bibliographies}
\label{sec:sectionbib}
% ==============================================================================

% ==============================================================================
\subsection{Structure}
% ==============================================================================

The thesis comprises of a main \TeX-files and \TeX-files for each chapter. Furthermore a \texttt{Makefile} can be used to gather the compilation instructions. The suggested structure is as follows
%
\begin{verbatim}
  main.tex
  example_chapter1.tex
  example_chapter2.tex
  ...
  library.bib
  figures/
\end{verbatim}
%

% ==============================================================================
\subsection{Main document}
% ==============================================================================

The layout is as follows
\begin{mdframed}
\begin{verbatim}
\documentclass[sectionbib,...]{goose-thesis}

\title{...}

\author{...}

...

\begin{document}

  \maketitle

  \setcounter{tocdepth}{0}
  \tableofcontents

  \cleardoublepage
  \include{example_chapter1}

  \cleardoublepage
  \include{example_chapter2}

  ...

\end{document}
\end{verbatim}
\end{mdframed}

Notice that in this case the main file does not have a \verb|\bibliography{...}| command. Rather, each of the chapter contains this command (unless of course that are no citations in that chapter). Also notice that the \verb|sectionbib| option has been used.

% ==============================================================================
\subsection{Chapters}
% ==============================================================================

The layout is as follows
\begin{mdframed}
\begin{verbatim}
\begin{chapter}

\begin{frontmatter}

\begin{abstract}
...
\end{abstract}

\keywords{...}

\begin{remark}
...
\end{remark}

\end{frontmatter}

...

\bibliography{...}

\appendix

...

\end{verbatim}
\end{mdframed}

Notice the \verb|\bibliography{...}| command at the end of the chapter.

% ==============================================================================
\subsection{Compilation}
% ==============================================================================

In this case the compilation is a bit more involved, as several bibliographies have to be created. For this example the steps are included in the following \texttt{Makefile}:

\begin{mdframed}
\begin{verbatim}
all:
  xelatex -interaction=nonstopmode example.tex
  xelatex -interaction=nonstopmode example.tex
  bibtex example_chapter1
  bibtex example_chapter2
  xelatex -interaction=nonstopmode example.tex
  xelatex -interaction=nonstopmode example.tex

clean:
  rm *.aux *.bbl *.log *.out *.pdf *.toc *.blg *.fls *.fdb_latexmk
\end{verbatim}
\end{mdframed}

% ==============================================================================
\section{Citations}
% ==============================================================================

Citations and references are handled using \texttt{natbib}. To cite use
\begin{verbatim}
  \citep{...}  (or \cite{...})
  \citet{...}
\end{verbatim}
The former only inserted a citation as number. For example \citep{geus}. The latter also includes the name(s) of the author(s). For example \citet{geus,wiki}.

The bibliography information is stored in a \texttt{bib}-file, which is included using
\begin{verbatim}
  \bibliography{...}
\end{verbatim}
This command creates a chapter section ``References'' with the bibliography. By default number citations are used, in which the references appear in the order in which they were cited. In the case that the \verb|namecite| option is used, the citations appear in names in alphabetical order.

Note that a large part of the formatting of Bib\TeX~depends on the formatting of the \texttt{bib}-file. A Python-script \texttt{bibparse} is available to automatically clean-up the formatting of the \texttt{bib}-file. An updated \texttt{unsrtnat.bst} is available that includes the \texttt{eprint} field.

% ==============================================================================
\bibliography{example_refs}
% ==============================================================================

\appendix

% ==============================================================================
\section{Some appendix}
% ==============================================================================


  \cleardoublepage
  
\chapter{Another chapter}

\begin{frontmatter}

\begin{abstract}
This is another chapter.
\end{abstract}

\end{frontmatter}

\section{With a section}

And a citation \citep{wiki}.

\bibliography{example_refs}

\appendix

\section{And an appendix}


  ...

\end{document}
\end{verbatim}
\end{mdframed}

Notice that in this case the main file does not have a \verb|\bibliography{...}| command. Rather, each of the chapter contains this command (unless of course that are no citations in that chapter). Also notice that the \verb|sectionbib| option has been used.

% ==============================================================================
\subsection{Chapters}
% ==============================================================================

The layout is as follows
\begin{mdframed}
\begin{verbatim}
\begin{chapter}

\begin{frontmatter}

\begin{abstract}
...
\end{abstract}

\keywords{...}

\begin{remark}
...
\end{remark}

\end{frontmatter}

...

\bibliography{...}

\appendix

...

\end{verbatim}
\end{mdframed}

Notice the \verb|\bibliography{...}| command at the end of the chapter.

% ==============================================================================
\subsection{Compilation}
% ==============================================================================

In this case the compilation is a bit more involved, as several bibliographies have to be created. For this example the steps are included in the following \texttt{Makefile}:

\begin{mdframed}
\begin{verbatim}
all:
  xelatex -interaction=nonstopmode example.tex
  xelatex -interaction=nonstopmode example.tex
  bibtex example_chapter1
  bibtex example_chapter2
  xelatex -interaction=nonstopmode example.tex
  xelatex -interaction=nonstopmode example.tex

clean:
  rm *.aux *.bbl *.log *.out *.pdf *.toc *.blg *.fls *.fdb_latexmk
\end{verbatim}
\end{mdframed}

% ==============================================================================
\section{Citations}
% ==============================================================================

Citations and references are handled using \texttt{natbib}. To cite use
\begin{verbatim}
  \citep{...}  (or \cite{...})
  \citet{...}
\end{verbatim}
The former only inserted a citation as number. For example \citep{geus}. The latter also includes the name(s) of the author(s). For example \citet{geus,wiki}.

The bibliography information is stored in a \texttt{bib}-file, which is included using
\begin{verbatim}
  \bibliography{...}
\end{verbatim}
This command creates a chapter section ``References'' with the bibliography. By default number citations are used, in which the references appear in the order in which they were cited. In the case that the \verb|namecite| option is used, the citations appear in names in alphabetical order.

Note that a large part of the formatting of Bib\TeX~depends on the formatting of the \texttt{bib}-file. A Python-script \texttt{bibparse} is available to automatically clean-up the formatting of the \texttt{bib}-file. An updated \texttt{unsrtnat.bst} is available that includes the \texttt{eprint} field.

% ==============================================================================
\bibliography{example_refs}
% ==============================================================================

\appendix

% ==============================================================================
\section{Some appendix}
% ==============================================================================


  \cleardoublepage
  
\chapter{Another chapter}

\begin{frontmatter}

\begin{abstract}
This is another chapter.
\end{abstract}

\end{frontmatter}

\section{With a section}

And a citation \citep{wiki}.

\bibliography{example_refs}

\appendix

\section{And an appendix}


  ...

\end{document}
\end{verbatim}
\end{mdframed}

Notice that in this case the main file does not have a \verb|\bibliography{...}| command. Rather, each of the chapter contains this command (unless of course that are no citations in that chapter). Also notice that the \verb|sectionbib| option has been used.

% ==============================================================================
\subsection{Chapters}
% ==============================================================================

The layout is as follows
\begin{mdframed}
\begin{verbatim}
\begin{chapter}

\begin{frontmatter}

\begin{abstract}
...
\end{abstract}

\keywords{...}

\begin{remark}
...
\end{remark}

\end{frontmatter}

...

\bibliography{...}

\appendix

...

\end{verbatim}
\end{mdframed}

Notice the \verb|\bibliography{...}| command at the end of the chapter.

% ==============================================================================
\subsection{Compilation}
% ==============================================================================

In this case the compilation is a bit more involved, as several bibliographies have to be created. For this example the steps are included in the following \texttt{Makefile}:

\begin{mdframed}
\begin{verbatim}
all:
  xelatex -interaction=nonstopmode example.tex
  xelatex -interaction=nonstopmode example.tex
  bibtex example_chapter1
  bibtex example_chapter2
  xelatex -interaction=nonstopmode example.tex
  xelatex -interaction=nonstopmode example.tex

clean:
  rm *.aux *.bbl *.log *.out *.pdf *.toc *.blg *.fls *.fdb_latexmk
\end{verbatim}
\end{mdframed}

% ==============================================================================
\section{Citations}
% ==============================================================================

Citations and references are handled using \texttt{natbib}. To cite use
\begin{verbatim}
  \citep{...}  (or \cite{...})
  \citet{...}
\end{verbatim}
The former only inserted a citation as number. For example \citep{geus}. The latter also includes the name(s) of the author(s). For example \citet{geus,wiki}.

The bibliography information is stored in a \texttt{bib}-file, which is included using
\begin{verbatim}
  \bibliography{...}
\end{verbatim}
This command creates a chapter section ``References'' with the bibliography. By default number citations are used, in which the references appear in the order in which they were cited. In the case that the \verb|namecite| option is used, the citations appear in names in alphabetical order.

Note that a large part of the formatting of Bib\TeX~depends on the formatting of the \texttt{bib}-file. A Python-script \texttt{bibparse} is available to automatically clean-up the formatting of the \texttt{bib}-file. An updated \texttt{unsrtnat.bst} is available that includes the \texttt{eprint} field.

% ==============================================================================
\bibliography{example_refs}
% ==============================================================================

\appendix

% ==============================================================================
\section{Some appendix}
% ==============================================================================


  \cleardoublepage
  
\chapter{Another chapter}

\begin{frontmatter}

\begin{abstract}
This is another chapter.
\end{abstract}

\end{frontmatter}

\section{With a section}

And a citation \citep{wiki}.

\bibliography{example_refs}

\appendix

\section{And an appendix}


  ...

\end{document}
\end{verbatim}
\end{mdframed}

Notice that in this case the main file does not have a \verb|\bibliography{...}| command. Rather, each of the chapter contains this command (unless of course that are no citations in that chapter). Also notice that the \verb|sectionbib| option has been used.

% ==============================================================================
\subsection{Chapters}
% ==============================================================================

The layout is as follows
\begin{mdframed}
\begin{verbatim}
\begin{chapter}

\begin{frontmatter}

\begin{abstract}
...
\end{abstract}

\keywords{...}

\begin{remark}
...
\end{remark}

\end{frontmatter}

...

\bibliography{...}

\appendix

...

\end{verbatim}
\end{mdframed}

Notice the \verb|\bibliography{...}| command at the end of the chapter.

% ==============================================================================
\subsection{Compilation}
% ==============================================================================

In this case the compilation is a bit more involved, as several bibliographies have to be created. For this example the steps are included in the following \texttt{Makefile}:

\begin{mdframed}
\begin{verbatim}
all:
  xelatex -interaction=nonstopmode example.tex
  xelatex -interaction=nonstopmode example.tex
  bibtex example_chapter1
  bibtex example_chapter2
  xelatex -interaction=nonstopmode example.tex
  xelatex -interaction=nonstopmode example.tex

clean:
  rm *.aux *.bbl *.log *.out *.pdf *.toc *.blg *.fls *.fdb_latexmk
\end{verbatim}
\end{mdframed}

% ==============================================================================
\section{Citations}
% ==============================================================================

Citations and references are handled using \texttt{natbib}. To cite use
\begin{verbatim}
  \citep{...}  (or \cite{...})
  \citet{...}
\end{verbatim}
The former only inserted a citation as number. For example \citep{geus}. The latter also includes the name(s) of the author(s). For example \citet{geus,wiki}.

The bibliography information is stored in a \texttt{bib}-file, which is included using
\begin{verbatim}
  \bibliography{...}
\end{verbatim}
This command creates a chapter section ``References'' with the bibliography. By default number citations are used, in which the references appear in the order in which they were cited. In the case that the \verb|namecite| option is used, the citations appear in names in alphabetical order.

Note that a large part of the formatting of Bib\TeX~depends on the formatting of the \texttt{bib}-file. A Python-script \texttt{bibparse} is available to automatically clean-up the formatting of the \texttt{bib}-file. An updated \texttt{unsrtnat.bst} is available that includes the \texttt{eprint} field.

% ==============================================================================
\bibliography{example_refs}
% ==============================================================================

\appendix

% ==============================================================================
\section{Some appendix}
% ==============================================================================
